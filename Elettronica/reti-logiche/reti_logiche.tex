\documentclass[a4paper,12pt]{article}

% Pacchetti per la matematica e simboli
\usepackage{amsmath}    % Formule matematiche
\usepackage{amssymb}    % Simboli matematici
\usepackage{amsfonts}   % Font matematici
\usepackage{mathrsfs}   % Scrittura calligrafica matematica
\usepackage{amsthm}     % Teoremi e definizioni

% Pacchetti per grafici
\usepackage{graphicx}   % Inclusione grafica
\usepackage{tikz}       % Disegni e grafici
\usepackage{pgfplots}   % Grafici avanzati
\pgfplotsset{compat=1.18} % Imposta compatibilità per pgfplots
\usetikzlibrary{circuits.logic.US}

% Pacchetti per la gestione degli indici e riferimenti
\usepackage{hyperref}   % Riferimenti ipertestuali
\hypersetup{
    colorlinks=true,    % Colori per i link
    linkcolor=darkblue, % Colore dei link interni
    citecolor=darkblue, % Colore dei riferimenti alle citazioni
    filecolor=darkblue, % Colore dei link ai file
    urlcolor=darkblue,  % Colore degli URL
    pdftitle={Formulario di Analisi Reale},
    pdfpagemode=UseOutlines
}
\usepackage{tocbibind}  % Include l'indice e la bibliografia nel sommario
\usepackage{fancyhdr}   % Intestazioni e piè di pagina personalizzati
\usepackage{bookmark}   % Gestione segnalibri PDF

% Configurazione per intestazioni e piè di pagina
\setlength{\headheight}{15pt} % Altezza dell'intestazione
\addtolength{\topmargin}{-10pt} % Riduzione del margine superiore

\pagestyle{fancy}
\fancyhf{}  % Pulisce intestazione e piè di pagina
\fancyhead[L]{Formulario di Analisi Reale}  % Intestazione sinistra
\fancyhead[R]{\today}                      % Intestazione destra
\fancyfoot[C]{\thepage}                    % Numero di pagina al centro

% Impostazioni dei margini
\usepackage{geometry}
\geometry{
    left=1.2in,
    right=1.2in,
    top=1in,
    bottom=1.2in
}

% Gestione delle sezioni e sottosezioni
\usepackage{titlesec}
\titleformat{\section}[block]{\large\scshape}{\thesection}{1em}{} % Titoli sezioni maiuscoli
\titleformat{\subsection}[block]{\normalsize\bfseries}{\thesubsection}{1em}{}
\titleformat{\subsubsection}[block]{\normalsize\itshape}{\thesubsubsection}{1em}{}
\titleformat{\paragraph}[runin]{\normalsize\bfseries}{\theparagraph}{1em}{}[:]
\titleformat{\subparagraph}[runin]{\normalsize\itshape}{\thesubparagraph}{1em}{}[:]

% Colore e aspetto
\usepackage{xcolor}
\definecolor{darkblue}{rgb}{0.0, 0.2, 0.6}
\definecolor{gray}{rgb}{0.5, 0.5, 0.5}

% Teoremi e definizioni formali
\newtheoremstyle{mystyle} % Definisci lo stile del teorema
  {10pt} % Spaziatura sopra
  {10pt} % Spaziatura sotto
  {\itshape} % Corpo del teorema in corsivo
  {} % Indentazione del numero
  {\bfseries} % Font del titolo
  {}     % Punteggiatura dopo il titolo
  {\newline} % Spazio dopo il titolo
  {}     % Stile del testo

\theoremstyle{mystyle}
\newtheorem{theorem}{Teorema}[section]
\newtheorem{definition}[theorem]{Definizione}
\newtheorem{lemma}[theorem]{Lemma}
\newtheorem{corollary}[theorem]{Corollario}
\newtheorem{proposition}[theorem]{Proposizione}

% Pacchetti aggiuntivi
\usepackage{enumitem}   % Gestione elenchi personalizzati
\usepackage{multicol}   % Colonne multiple
\usepackage{booktabs}   % Tabelle formattate professionalmente
\usepackage{caption}    % Migliore gestione delle didascalie
\captionsetup{
    labelfont=bf,
    font=small,
    labelsep=colon
}

% Impostazioni di line spacing
\usepackage{setspace}
\onehalfspacing  % Interlinea 1.5 per migliorare la leggibilità

% Informazioni del documento
\title{\textbf{Reti Logiche}}
\author{\textit{Oudeys}}
\date{\today}

% Inizio del documento
\begin{document}

\maketitle



\tableofcontents
\newpage

\section{Introduzione}
\begin{definition}[Circuito]
    Un \textbf{circuito} è un insieme di componenti elettronici interconnessi che operano insieme per eseguire una determinata funzione. I circuiti possono essere combinatori, dove l'uscita dipende solo dagli ingressi correnti, o sequenziali, dove l'uscita dipende anche dagli stati precedenti (memoria). 
    \begin{itemize}
        \item \textbf{Circuito combinatorio}: L'uscita è una funzione puramente degli ingressi presenti.
        \item \textbf{Circuito sequenziale}: L'uscita dipende sia dagli ingressi presenti che dalla storia degli ingressi precedenti, tramite l'uso di elementi di memoria.
    \end{itemize}
\end{definition}

\section{Circuiti combinatori}
\subsection{Porte logiche}

\subsubsection{Operatori base}

\paragraph{AND}
\[
\begin{array}{|c|c|c|}
\hline
x & y & x \land y \\
\hline
0 & 0 & 0 \\
0 & 1 & 0 \\
1 & 0 & 0 \\
1 & 1 & 1 \\
\hline
\end{array}
\]



\paragraph{OR}
\[
\begin{array}{|c|c|c|}
\hline
x & y & x \lor y \\
\hline
0 & 0 & 0 \\
0 & 1 & 1 \\
1 & 0 & 1 \\
1 & 1 & 1 \\
\hline
\end{array}
\]

\paragraph{NOT}
\[
\begin{array}{|c|c|}
\hline
x & \lnot x \\
\hline
0 & 1 \\
1 & 0 \\
\hline
\end{array}
\]

\subsubsection{Operatori completi}
\paragraph{NAND}
\[
\begin{array}{|c|c|c|}
\hline
x & y & \lnot(x \land y) \\
\hline
0 & 0 & 1 \\
0 & 1 & 1 \\
1 & 0 & 1 \\
1 & 1 & 0 \\
\hline
\end{array}
\]

\paragraph{NOR}
\[
\begin{array}{|c|c|c|}
\hline
x & y & \lnot(x \lor y) \\
\hline
0 & 0 & 1 \\
0 & 1 & 0 \\
1 & 0 & 0 \\
1 & 1 & 0 \\
\hline
\end{array}
\]

\paragraph{XOR}
\[
\begin{array}{|c|c|c|}
\hline
x & y & x \oplus y \\
\hline
0 & 0 & 0 \\
0 & 1 & 1 \\
1 & 0 & 1 \\
1 & 1 & 0 \\
\hline
\end{array}
\]

\paragraph{XNOR}
\[
\begin{array}{|c|c|c|}
\hline
x & y & \lnot(x \oplus y) \\
\hline
0 & 0 & 1 \\
0 & 1 & 0 \\
1 & 0 & 0 \\
1 & 1 & 1 \\
\hline
\end{array}
\]

\subsection{Algebra di Boole}
\subsubsection{Assiomi dell'algebra booleana}


\begin{definition}[Identità]
    \[
        x+0=x
    \]
    \[
        y \bullet 1 = y
    \]
\end{definition}

\begin{definition}[Commutativa]
    \[
        x+y=y+x
    \]
    \[
        x \bullet y = y \bullet x
    \]
\end{definition}

\begin{definition}[Distributiva]
    \[
        x \bullet (y+z) = (x \bullet y)+(x \bullet z)
    \]
    \[
        x+(y \bullet z) = (x+y) \bullet (x+z)
    \]
\end{definition}

\begin{definition}[Complementazione]
    \[
        x+x'=1
    \]
    \[
        x \bullet x' = 0
    \]
\end{definition}

\subsubsection{Teoremi dell'algebra booleana}
\begin{proposition}[Proprietà associativa]
    \[
        x+(y+z) = (x+y) +z
    \]
    \[
        x(yz)=(xy)z
    \]
\end{proposition}

\begin{proposition}[Legge dell'elemento nullo]
    \[
        x+1=1
    \]
    \[
        x \bullet 0 =0
    \]
\end{proposition}

\begin{proposition}[Involuzione]
    \[
        (x')'=x
    \]
\end{proposition}

\begin{proposition}[Indepotenza]
    \[
        x+x=x
    \]
    \[
        x \bullet x = x
    \]
\end{proposition}

\begin{proposition}[Assorbimento]
    \[
        x+xy=x
    \]
    \[
        x(x+y)=x
    \]
\end{proposition}

\begin{proposition}[Semplificazione]
    \[
        x+x'y=x+y
    \]
    \[
        x(x'+y)=xy
    \]
\end{proposition}

\begin{proposition}[Adiacenza]
    \[
        xy + xy' = x
    \]
    \[
        (x+y)(x+y')=x
    \]
\end{proposition}

\begin{theorem}[Leggi di De Morgan]
    \[
        (x+y)' = x' \bullet y'
    \]
    \[
        (x'+y') = x \bullet y
    \]

    \[  
        (x \bullet y)' = x' + y'
    \]
    \[
        (x' \bullet y')' = x+y
    \]
\end{theorem}

\subsubsection{Teorema di espansione di Shannon}

\begin{theorem}[Teorema di espansione di Shannon]
    \[
        f(x_1, \ldots , x_n) = x_1 \cdot f(1,x_2, \ldots, x_n) + x'_1 \cdot f (0, x_2, \ldots, x_n)
    \]
\end{theorem}

\paragraph{Forme canoniche}

\begin{definition}[Somma di prodotti]
    \[
        f(a,b,c) = a'b'c' + a'b'c +a'bc' + ab'c + abc
    \]
\end{definition}

\begin{definition}[Prodotto di somme]
    \[
        f(a,b,c) = (a+b'+c') \bullet (a'+b+c) \bullet (a'+b'+c)
    \]
\end{definition}

\subsection{Sintesi a due livelli}
\begin{definition}[Livello]
    \[
        \text{Massimo numero di porte logihe attraversate dall'ingresso all'uscita.}
    \]
\end{definition}

\begin{definition}[Letterale]
    \[
        \text{Variabile in forma affermata o in forma negata.}
    \]
\end{definition}

\begin{definition}[Minterm - Prodotto fondamentale]
    \[
        \text{Prodotto in cui ogni variabile compare una volta come letterale.}
    \]  
\end{definition}

\begin{definition}[Maxterm - Somma fondamentale]
    \[
        \text{Somma in cui ogni variabile compare una volta come letterale.}
    \]
\end{definition}

\begin{definition}[Implicante]
    Siano \( f \) e \( g \) funzioni di \( n \) variabili.
    \[
        g \text{ è implicante di } f \Leftrightarrow \text{per qualunque assegnamento } (x_1, \ldots, x_n) \text{ alle variabili:}
    \]
    \[
        g(x_1, \ldots, x_n) = 1 \Rightarrow f(x_1, \ldots, x_n) = 1
    \]
    Quando \( g \) è implicante di \( f \), possiamo affermare:
    \begin{enumerate}[label=(\roman*)]
        \item Se \( g \) vale 1 in qualche punto, allora \( f \) vale 1.
        \item Se \( g \) vale 0, allora \( f \) può essere indifferentemente 0 o 1.
    \end{enumerate}
\end{definition}

\begin{definition}[Term]
    \[
        \text{Gruppo di 1 di \(f\) che non sia un minterm.}
    \]
\end{definition}

\begin{definition}[Multiplexer]
    Un \textbf{multiplexer} (o MUX) è un circuito combinatorio che seleziona una delle molteplici linee di ingresso e la indirizza a un'unica uscita. Il multiplexer utilizza \( n \) linee di selezione per scegliere quale tra \( 2^n \) ingressi indirizzare all'uscita.
    \[
        Y = D_i \quad \text{dove} \quad i \text{ è determinato dalle linee di selezione.}
    \]
    Il multiplexer è spesso utilizzato per ridurre la complessità di un sistema, consentendo il controllo di più segnali con un numero ridotto di risorse di controllo.
\end{definition}



\begin{definition}[Mappa di Karnaugh]
    La \textbf{Mappa di Karnaugh} (K-map) è una rappresentazione grafica utilizzata per semplificare funzioni logiche booleane. Organizza i termini minterm di una funzione in una griglia, dove le posizioni adiacenti differiscono per un solo bit, permettendo di visualizzare chiaramente le ridondanze e le semplificazioni.
    Ogni cella della mappa corrisponde a un minterm, e la semplificazione avviene raggruppando celle adiacenti.
    La Mappa di Karnaugh facilita la riduzione di espressioni logiche per minimizzare il numero di porte logiche necessarie in un circuito digitale.
\end{definition}

\begin{definition}[Implicanti primi]
    Un implicante è primo quando non è contenuto in un altro implicante.
    \begin{enumerate}
        \item Ridondanti: uno degli implicanti non serve perché gli altri da soli già coprono la funzione.
        \item Essenziali: coprono un 1 non coperto da nessun altro implicante primo, devono necessariamente comparire in qualunque copertura minima.
    \end{enumerate}
\end{definition}

\begin{definition}[Semplificazione]
    \begin{enumerate}
        \item Trovare il più piccolo insieme di implicanti primi che ricoprano tutta la funzione.
        \item Scrivere la tabella della verità in forma di mappa di Karnaugh.
        \item Partendo dai minterm ci si espande per trovare gli implicanti primi.
        \item Si scrive l'espressione degli implicanti guardando le variabili che sono costanti.
        \item Si sceglie un insieme di implicanti non ridondante.
    \end{enumerate}
\end{definition}

\begin{definition}[Priority encoder]
    
\end{definition}

\subsection{Circuiti combinatori di base}

\subsection{Diagrammi temporali}

\section{Circuiti aritmetici}

\section{Circuiti sequenziali}



\end{document}