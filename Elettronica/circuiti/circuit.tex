\documentclass[a4paper,12pt]{article}

% Pacchetti per la matematica e simboli
\usepackage{amsmath}    % Formule matematiche
\usepackage{amssymb}    % Simboli matematici
\usepackage{amsfonts}   % Font matematici
\usepackage{mathrsfs}   % Scrittura calligrafica matematica
\usepackage{amsthm}     % Teoremi e definizioni

% Pacchetti per grafici
\usepackage{graphicx}   % Inclusione grafica
\usepackage{tikz}       % Disegni e grafici
\usepackage{pgfplots}   % Grafici avanzati
\pgfplotsset{compat=1.18} % Imposta compatibilità per pgfplots
\usepackage{circuitikz}

% Pacchetti per la gestione degli indici e riferimenti
\usepackage{hyperref}   % Riferimenti ipertestuali
\hypersetup{
    colorlinks=true,    % Colori per i link
    linkcolor=darkblue, % Colore dei link interni
    citecolor=darkblue, % Colore dei riferimenti alle citazioni
    filecolor=darkblue, % Colore dei link ai file
    urlcolor=darkblue,  % Colore degli URL
    pdftitle={Formulario di Analisi Reale},
    pdfpagemode=UseOutlines
}
\usepackage{tocbibind}  % Include l'indice e la bibliografia nel sommario
\usepackage{fancyhdr}   % Intestazioni e piè di pagina personalizzati
\usepackage{bookmark}   % Gestione segnalibri PDF

% Configurazione per intestazioni e piè di pagina
\setlength{\headheight}{15pt} % Altezza dell'intestazione
\addtolength{\topmargin}{-10pt} % Riduzione del margine superiore

\pagestyle{fancy}
\fancyhf{}  % Pulisce intestazione e piè di pagina
\fancyhead[L]{Formulario di Analisi Reale}  % Intestazione sinistra
\fancyhead[R]{\today}                      % Intestazione destra
\fancyfoot[C]{\thepage}                    % Numero di pagina al centro

% Impostazioni dei margini
\usepackage{geometry}
\geometry{
    left=0.2in,
    right=0.2in,
    top=1in,
    bottom=0.5in
}

% Gestione delle sezioni e sottosezioni
\usepackage{titlesec}
\titleformat{\section}[block]{\large\scshape}{\thesection}{1em}{} % Titoli sezioni maiuscoli
\titleformat{\subsection}[block]{\normalsize\bfseries}{\thesubsection}{1em}{}
\titleformat{\subsubsection}[block]{\normalsize\itshape}{\thesubsubsection}{1em}{}
\titleformat{\paragraph}[runin]{\normalsize\bfseries}{\theparagraph}{1em}{}[:]
\titleformat{\subparagraph}[runin]{\normalsize\itshape}{\thesubparagraph}{1em}{}[:]

% Colore e aspetto
\usepackage{xcolor}
\definecolor{darkblue}{rgb}{0.0, 0.2, 0.6}
\definecolor{gray}{rgb}{0.5, 0.5, 0.5}

% Teoremi e definizioni formali
\newtheoremstyle{mystyle} % Definisci lo stile del teorema
  {10pt} % Spaziatura sopra
  {10pt} % Spaziatura sotto
  {\itshape} % Corpo del teorema in corsivo
  {} % Indentazione del numero
  {\bfseries} % Font del titolo
  {}     % Punteggiatura dopo il titolo
  {\newline} % Spazio dopo il titolo
  {}     % Stile del testo

\theoremstyle{mystyle}
\newtheorem{theorem}{Teorema}[section]
\newtheorem{definition}[theorem]{Definizione}
\newtheorem{lemma}[theorem]{Lemma}
\newtheorem{corollary}[theorem]{Corollario}
\newtheorem{proposition}[theorem]{Proposizione}

% Pacchetti aggiuntivi
\usepackage{enumitem}   % Gestione elenchi personalizzati
\usepackage{multicol}   % Colonne multiple
\usepackage{booktabs}   % Tabelle formattate professionalmente
\usepackage{caption}    % Migliore gestione delle didascalie
\captionsetup{
    labelfont=bf,
    font=small,
    labelsep=colon
}

% Impostazioni di line spacing
\usepackage{setspace}
\onehalfspacing  % Interlinea 1.5 per migliorare la leggibilità

% Informazioni del documento
\title{\textbf{Teoria dei circuiti}}
\author{\textit{Oudeys}}
\date{\today}

% Inizio del documento
\begin{document}

\maketitle

\newpage


\tableofcontents
\newpage

%\begin{multicols}{2}
    

\section{Circuiti a parametri concentrati}

\subsection{Leggi di Kirchhoff}

\subsubsection{Legge delle correnti di Kirchhoff}

\begin{definition}[Legge delle correnti]
    \[
        \sum_{k=0}^N i_k (t) = 0
    \]
\end{definition}

\subsubsection{Legge delle tensioni di Kirchhoff}

\begin{definition}[Legge delle tensioni]
    \[
        \sum_{k=0}^N V_i = 0
    \]
\end{definition}

\section{Elementi circuitali}
\subsection{Resistori}

\subsubsection{Resistori lineari}
\begin{definition}[Resistori lineari tempo invarianti]
    \[\newline\]
    \begin{enumerate}[label=\roman*.]
        \item \[v(t) = R i(t)\]
        \item \[i(t) = Gv(t)\]
        \item \[R = 1/G\]
    \end{enumerate}
\end{definition}

\begin{definition}[Resistori lineari tempo varianti]
    \[\newline\]
    \begin{enumerate}[label=\roman*.]
        \item \[v(t)=R(t)i(t)\]
        \item \[i(t)=G(t)v(t)\]
        \item \[R(t)=1/G(t)\]
    \end{enumerate}
\end{definition}

\subsubsection{Resistori non lineari}

\begin{definition}[Resistori non lineari tempo invarianti]
    \[\newline\]
    \begin{enumerate}[label=\roman*.]
        \item Controllato in corrennte \[v(t)=f(i(t))\]
        \item Controllato in tensione \[i(t)=g(v(t))\]
    \end{enumerate}    
\end{definition}

\begin{definition}[Resistori non lineari tempo varianti]
    \[\newline\]
    \begin{enumerate}[label=\roman*.]
        \item Controllato in corrente \[v(t)=f(i(t),t)\]
        \item Controllato in tensione \[i(t)=g(v(t),t)\]
    \end{enumerate}
\end{definition}

\subsection{Generatori indipendenti}
\subsubsection{Generatore di tensione}
\subsubsection{Generatore di corrente}
\subsubsection{Circuiti equivalenti di Thèvenin e Norton}
\subsubsection{Forme d'onda}

\subsection{Condensatori}
\subsubsection{Condensatori lineari}
\begin{definition}[Condensatori lineari tempo invarianti]
    \[\newline\]
    \begin{enumerate}[label=\roman*.]
        \item \[q(t)=Cv(t)\]
        \item \[i(t)= C \frac{dv}{dx}\]
        \item \[v(t)=v(0)+\frac{1}{C} \int_{0}^{t} i(t') dt'\]
    \end{enumerate}
\end{definition}

\begin{definition}[Condensatori lineari tempo varianti]
    \[\newline\]
    \begin{enumerate}[label=\roman*.]
        \item \[q(t)=C(t)v(t)\]
        \item \[i(t)= \frac{dC}{dt} v(t) + C(t) \frac{dv}{dx}\]
    \end{enumerate}
\end{definition}

\subsubsection{Condensatori non lineari}
\begin{definition}[Condensatori non lineari tempo invarianti]
    \[\newline\]
    \begin{enumerate}[label=\roman*.]
        \item \[q(t)=f(v(t))\]
        \item \[i(t) = \frac{df}{dv} \bigg |_{v(t)} \frac{dv}{dt}\]
    \end{enumerate}
\end{definition}

\begin{definition}[Condensatori non lineari tempo varianti]
    \[\newline\]
    \begin{enumerate}[label=\roman*.]
        \item \[q(t)=f(v(t),t)\]
        \item \[i(t) = \frac{\partial f}{\partial t} \bigg |_{v(t)} \frac{dv}{dt}\]
    \end{enumerate}
\end{definition}

\subsection{Induttori}

\subsubsection{Induttori lineari}
\begin{definition}[Induttori lineari tempo invarianti]
    \[\newline\]
    \begin{enumerate}[label=\roman*.]
        \item \[\phi = L i(T)\]
        \item \[v(t)= L \frac{di}{dt}\]
        \item \[i(t)=i(0)+\frac{1}{L}\int_0^t v(t') dt'\]
    \end{enumerate}
\end{definition}

\begin{definition}[Induttori lineari tempo varianti]
    \[\newline\]
    \begin{enumerate}[label=\roman*.]
        \item \[\phi = L(t)i(t)\]
        \item \[v(t)=\frac{dL}{dt}i(t)+L(t) \frac{di}{dt}\]
    \end{enumerate}
\end{definition}

\subsubsection{Induttori non lineari}
\begin{definition}[Induttori non lineari tempo invarianti]
    \[\newline\]
    \begin{enumerate}[label=\roman*.]
        \item \[\phi(t)=f(i(t))\]
        \item \[v(t)= \frac{df}{di} \bigg |_{i(t)} \frac{di}{dt}\]
    \end{enumerate}
\end{definition}

\begin{definition}[Induttori non lineari tempo varianti]
    \[\newline\]
    \begin{enumerate}[label=\roman*.]
        \item \[\phi(t)=f(i(t),t)\]
        \item \[v(t)= \frac{\partial f}{\partial t} + \frac{\partial f}{\partial i} \bigg |_{i(t)} \frac{di}{dt}\]
    \end{enumerate}
\end{definition}

\section{Circuiti semplici}

\subsection{Resistori}

\begin{enumerate}[label=\roman*.]
    \item Serie \[R = \sum_{k=1}^{m} R_k\]
    \item Parallelo \[G = \sum_{k=1}^{m} G_k\]
\end{enumerate}

\subsection{Condensatori}

\begin{enumerate}[label=\roman*.]
    \item Serie \[S = \sum_{k=1}^{m} S_k\]
    \item Parallelo \[C = \sum_{k=1}^{m} C_k\]
\end{enumerate}

\subsection{Induttori}

\begin{enumerate}
    \item Serie \[L = \sum_{k=1}^{m} L_k\]
    \item Parallelo \[\Gamma = \sum_{k=1}^{m} \Gamma_k\]
\end{enumerate}

\section{Circuiti del I ordine}

\subsection{Risposta con ingresso zero}
\subsubsection{Circuito RC (Resistore-Condensatore)}


\subsubsection{Circuito RL (Resistore-Induttore)}

\subsection{Risposta con stato zero}

\subsubsection{Ingresso corrente costante}

\subsubsection{Ingresso sinusoidale}

\subsection{Risposta completa}

\subsubsection{Risposta completa}

\subsubsection{Transitorio e regime}

\subsubsection{Circuiti con due costanti di tempo}

\subsection{Linearità della risposta con stato zero}

\subsection{Linearità ed invarianza temporale}
\subsubsection{Risposta al gradino}

\subsubsection{Invarianza temporale}

\subsubsection{Traslazione}

\subsection{Risposta all'impulso}

\subsection{Risposta al gradino e all'impulso per circuiti semplici}



\section{Circuiti del II ordine}
\subsection{Risposta con ingresso zero}

\subsection{Risposta con stato zero}

\subsubsection{Risposta al gradino}

\subsection{Spazio degli stati}

\subsubsection{Equazioni di stato e traiettoria}
\subsubsection{Rappresentazione matriciale}
\subsubsection{Metodo approssimato per il calcolo della traiettoria}
\subsubsection{Equazioni di stato e risposta completa}

\section{Circuiti lineari tempo invarianti}

\subsection{Analisi dei nodi e delle maglie}

\subsubsection{Analisi dei nodi}

\subsubsection{Analisi delle maglie}

\subsection{Rappresentazione ingresso-uscita}
\subsubsection{Risposta con ingresso zero}
\subsubsection{Risposta con stato zero}
\subsubsection{Risposta all'impulso}
\subsection{Risposta ad un ingresso arbitrario}
\subsubsection{Integrale di convoluzione}

\section{Analisi in regime sinusoidale}
\subsection{Risposta completa e risposta in regime sinusoidale}
\subsubsection{Risposta completa}
\subsubsection{Risposta in regime sinusoidale}
\subsubsection{Sovrapposizione nel regime stazionario}

\subsection{Impedenza e ammettenza}

\subsection{Analisi in regime sinusoidale dei circuiti semplici}
\subsubsection{Collegamenti in serie e in parallelo}
\subsubsection{Analisi dei nodi e delle maglie in regime sinusoidale}
\subsection{Circuiti risonanti}

\subsection{Potenza in regime sinusoidale}

\subsection{Normalizzazione della frequenza e della impedenza}

\section{Elementi di accoppiamento e circuiti accoppiati}
\subsection{Induttori accoppiati}
\subsection{Trasformatori ideali}
\subsection{Generatori pilotati}

\section{Grafi delle reti}

\section{Teorema di Tellegen}

\section{Analisi dei nodi e degli anelli}
\subsection{Trasformazioni di generatori}
\subsection{Analisi dei nodi delle reti lineari tempo-invarianti}
\subsubsection{Analisi di reti sesistive}
\subsubsection{Formulazione rapida delle equazioni dei nodi}
\subsubsection{Analisi in regime sinusoidale}
\subsubsection{Equazioni integrodifferenziali}

\subsection{Dualità}

\section{Analisi delle maglie e degli insiemi di taglio}
\subsection{Teorema fondamentale della teoria dei grafi}
\subsection{Analisi delle maglie}
\subsection{Analisi degli insiemi di taglio}

\section{Equazioni di stato}
\subsection{Reti lineari tempo-invarianti}
\subsection{Concetto di stato}
\subsection{Reti non lineari e tempo-invarianti}
\subsubsection{Caso lineare tempo-variante}
\subsubsection{Caso non lineare}
\subsubsection{Equazioni di stato per reti lineari tempo-invarianti}

\newpage
\section{Trasformate di Laplace}
\subsection{Trasformata di Laplace}
\begin{definition}[Trasformata di Laplace]
    \[\mathcal{L}[f(t)] = \int_{0^-}^{\infty} f(t) e^{-st} dt\]
\end{definition}

\subsection{Proprietà fondamentali della trasformata di Laplace}

\begin{proposition}[Proprietà fondamentali della Trasformata di Laplace]
    \[\newline\]
    \begin{enumerate}[label=\roman*.]
        \item Unicità \[F(s) = \mathcal{L} [f(t)] \Leftrightarrow f(t) = \mathcal{L}^{-1} [F(s)]\]
        \item Linearità \[\mathcal{L}[c_1 f_1(t) + c_2 f_2(t)] = c_1 \mathcal{L} [f_1(t)] + c_2 \mathcal{L} [f_2(t)]\]
        \item Differenziazione \[\mathcal{L} \left[\frac{df}{dt}\right] = s \mathcal{L} [f(t)] - f(0^-)\]
        \item Integrazione \[\mathcal{L} \left[\int_{0^-}^{t} f(t') dt'\right] = \frac{1}{s} \mathcal{L} [f(t)]\]
    \end{enumerate}
\end{proposition}

\subsection{Trasformate di Laplace di funzioni elementari}

\begin{proposition}
    \[\newline\]
    \begin{enumerate}[label=\roman*.]
        \item \[f(t) \leftrightarrow F(s) = \int_{0^-}^{\infty} f(t) e^{-st} dt\]
        \item \[\delta (t) \leftrightarrow 1\]
        \item \(\forall n \in \mathbb N\) \[\delta ^n (t) \leftrightarrow s^n\]
        \item \[u(t) \leftrightarrow \frac{1}{s}\]
        \item \(\forall n \in \mathbb N\) \[ \frac{t^n}{n!} \leftrightarrow \frac{1}{s^{n+1}}\]
        \item \[e^{-at} \leftrightarrow \frac{1}{s+a}\]
        \item \(\forall n \in \mathbb N\) \[\frac{t^n}{n!} e^{-at} \leftrightarrow \frac{1}{(s+a)^{n+1}}\]
        \item \[\cos (\beta t) \leftrightarrow \frac{s}{s^2 + \beta ^2}\]
        \item \[\sin (\beta t) \nLeftrightarrow \frac{\beta}{s^2 + \beta ^2}\]
        \item \[e^{-\alpha t} \cos (\beta t) \leftrightarrow \frac{s+ \alpha}{(s+ \alpha)^2 + \beta ^2}\]
        \item \[e^{-\alpha t} \sin (\beta t) \leftrightarrow \frac{\beta}{(s+ \alpha)^2 + \beta ^2}\]
        \item \[a e^{-\alpha t} \cos (\beta t) + \frac{(b - a \alpha)}{\beta} e^{-\alpha t} \sin (\beta t) \leftrightarrow \frac{as + b}{(s+\alpha)^2 + \beta ^"}\]
        \item \[2 \lvert k \rvert e^{-\alpha t} \cos(\beta t + \measuredangle k) \leftrightarrow \frac{k}{s+ \alpha - j\beta} + \frac{\overline k}{s+ \alpha + j \beta}\]
    \end{enumerate}
\end{proposition}

\subsection{Soluzione di circuiti semplici}
\subsubsection{Calcolo di una risposta all'impulso}
\begin{proposition}[Risposta all'impulso per circuiti RLC]
    \[\frac{L}{R} \frac{d v}{dt} + v + \frac{1}{RC} \int_{0^-}^{t} v(t') dt' + v_c (0^-) = e(t)\]
    \[\frac{L}{R} \frac{d h}{dt} + h + \frac{1}{RC} \int_{0^-}^{t} h(t') dt' = \delta(t)\]
    \[\frac{L}{R} \mathcal{L} \left[\frac{dh}{dt}\right] + H(s) + \frac{1}{RC} \mathcal{L} \left[ \int_{0^-}^{t} h(t') dt'\right] = 1\]
    \[\left[ \frac{L}{R} s + 1 + \frac{1}{RCs} \right] H(s) = 1 \Leftrightarrow H(s) = \frac{R}{L} \frac{s}{s^2 + (R/L)s + 1/LC}\]
    \[H(s) = \frac{R}{L} \frac{s}{(s+ \alpha)^2 + \omega _d ^2}\]
    \[h(t) = \frac{\omega _0 R}{\omega _d L} u(t) e^{-\alpha t} \cos (\omega _d t + \phi)\]
    \[\frac{V}{E} = \frac{R}{L} \frac{j \omega}{(j \omega)^2 + (R/L) j \omega + 1/LC}\]
    \[\frac{V}{E} = H(j \omega)\]
\end{proposition}

\subsubsection{Espansione in frazioni parziali}

\subsection{Soluzione di reti generali}
\subsubsection{Formulazione di equazioni lineari algebriche}
\subsubsection{Metodo del cofattore}
\subsubsection{Funzioni di rete e regime sinusoidale}
\subsubsection{Proprietà fondamentali delle reti lineari tempo-invarianti}

\subsection{Reti degeneri}
\subsection{Condizioni sufficienti per l'unicità}

\section{Frequenze naturali}
\subsection{Frequenza naturale di una variabile di rete}
\subsection{Metodo di eliminazione}
\subsection{Frequenze naturali di una rete}
\subsection{Frequenze naturali ed equazioni di stato}

\section{Funzioni di rete}
\subsection{Definizione e proprietà generali}
\subsection{Poli, zeri e risposta in frequenza}
\subsection{Poli, zeri e risposta all'impulso}
\subsection{Proprietà di simmetria}

\section{Teoremi delle reti}
\subsection{Teorema di sostituzione}
\subsection{Teorema di sovrapposizione}
\subsection{Teorema delle reti equivalenti di Thévenin e Norton}
\subsection{Teorema di reciprocità}

\section{Doppi bipoli}
\subsection{Doppi bipoli resistivi}
\subsection{Transistore}
\subsection{Induttori accoppiati}
\subsection{Matrici di impedenza ed ammettenza dei doppi bipoli}
\subsection{Matrici ibride}
\subsection{Matrici di trasmissione}

\section{Reti resistive}
\subsection{Reti fisiche e modelli di reti}
\subsection{Analisi delle reti resistive dal punto di vista della potenza}
\subsection{Guadagno di tensione e guadagno di corrente di rete resistiva}


\section{Energia e passività}
\subsection{Condensatore lineare tempo-variante}
\subsection{Energia immagazzinata in elementi non lineari tempo-varianti}
\subsection{Bipoli passivi}
\subsection{Ingresso esponenziale e risposta esponenziale}
\subsection{Bipoli costituiti di elementi passivi lineari tempo-invarianti}
\subsection{Stabilità delle reti passive}
\subsection{Amplificatore parametrico}


%\end{multicols}
\end{document}