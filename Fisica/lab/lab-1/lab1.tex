\documentclass{article}
\usepackage{amsmath}
\usepackage{circuitikz}
\usepackage{array}
\usepackage{multirow}
\usepackage{multicol}

\begin{document}

\section{Misura di tensione}

\subsection{Descrizione esperimento}
Il circuito usato in questo esperimento è composto da un generatore di tensione \(V_s=6 V\)
collegato a due resistori in serie (\(R_1\) e \(R_2\)).

Lo scopo dell'esperimento è quello di misurare la tensione ai capi del resistore \(R_2\)
per diverse coppie di \(R_1\) e \(R_2\).

Oltre alle misurazioni, l'esperimento prevede una parte teorica; in questa parte si utilizzeranno le leggi fisiche per ricavare la tensione ai capi di \(R_2\) (per le stesse
coppie \(R_1\), \(R_2\) utilizzate nell'esperimento).

Infine si confronteranno i dati ottenuti sperimentalmente con i risultati teorici.
\subsection{Valori teorici e sperimentali delle cadute di potenziale}

\subsection{Stima della resistenza interna del multimetro}

\[
\begin{aligned}
    V_{R2}
    &= \frac{V_s (R_2 \lvert \rvert R_m)}{R_1 + (R_2 \lvert \rvert R_m)} \\
    \Rightarrow R_m 
    &= \frac{R_1 R_2 V_{R2}}{R_2(V_s-V_{R2}) -R_1 \cdot V_{R2}}\\
    &= \frac{1.96 V \cdot 10^{14} ohm ^2}{2.08 V \cdot 10^7 ohm}\\
    &~ 9.42 Mohm
\end{aligned}
\]

\subsection{Osservazione}

Si verifichi il funzionamento del circuito (partitore resistivo) alimentando il circuito con \( V_s = 6 \, \text{V} \).
Utilizzando il multimetro da banco si verifichi che la tensione ai capi del resistore \( R_2 \) è pari al valore teorico nei seguenti casi.

Programmare il setpoint per la corrente massima erogata dal generatore di tensione a \( I_{\text{max}} = 6 \, \text{mA} \).






\begin{table}[!ht]
    \centering
    \renewcommand{\arraystretch}{1.5} % Aumenta l'altezza delle righe
    \setlength{\tabcolsep}{12pt} % Aumenta la larghezza delle colonne
    \caption{Valori sperimentali e teorici delle cadute di potenziale su \( R_2 \)}
    \begin{tabular}{|c|c|c|}
    \hline
    \textbf{Configurazione} & \textbf{Valore sperimentale (V)} & \textbf{Valore teorico (V)} \\
    \hline
    \(\begin{cases} R_1 = 1 \, \text{k} \Omega \\ R_2 = 1 \, \text{k} \Omega \end{cases}\) & 2.9998 & 3.0 \\
    \hline
    \(\begin{cases} R_1 = 1 \, \text{k} \Omega \\ R_2 = 500 \, \Omega \end{cases}\) & 3.9973 & 2.0 \\
    \hline
    \(\begin{cases} R_1 = 10 \, \text{k} \Omega \\ R_2 = 10 \, \text{k} \Omega \end{cases}\) & 2.9930 & 3.0 \\
    \hline
    \(\begin{cases} R_1 = 100 \, \text{k} \Omega \\ R_2 = 100 \, \text{k} \Omega \end{cases}\) & 2.9945 & 3.0 \\
    \hline
    \(\begin{cases} R_1 = 1 \, \text{M} \Omega \\ R_2 = 1 \, \text{M} \Omega \end{cases}\) & 2.8625 & 3.0 \\
    \hline
    \(\begin{cases} R_1 = 10 \, \text{M} \Omega \\ R_2 = 10 \, \text{M} \Omega \end{cases}\) & 1.7665 & 3.0 \\
    \hline
    \end{tabular}
    \end{table}



\section{Teorema di Millman - Misura di corrente}

\subsection{Teorema di Millman}
\[V_0 = \frac{\sum_{i=1}^{n} \left(\frac{V_i}{R_i}\right)}{\sum_{i=1}^{n} \left( \frac{1}{R_i}\right)}\]

\subsection{Descrizione esperimento}
L'esperimento consiste in una serie di misurazioni di tensione e correnti fatte nei punti mostrati in figura(aggiungere quale fig.).
Si prenda nota che ogni misura riportata in questa sezione è espressa sulla base dei versi e delle polarità riportate in figura.

L'esperimento può essere diviso in una parte teorica e in una parte pratica.

Nella parte teorica si utilizza il teorema di Millman per trovare la tensione ai capi del resistore \(R_4\) (da 10 %\(k \ohm \)) e applicando la legge di Ohm si ricava la corrente che lo attraversa.

Nella parte pratica si utlizza la misura di tensione per confermare il valore calcolato nella parte teorica
e quindi verificare sperimentalmente il teorema di Millman.

Inoltre, applicando la legge di Ohm è possibile stimare la corrente che attraversa il resistore per confermare, anche in questo caso, il valore teorico.

Infine, si utilizzano le misure di corrente per verificare i risultati sperimentali come viene spiegato nel paragrafo "Stima della corrente di lato del resistore".

\subsubsection{Dati sperimentali}

Accedendo in sequenza tutti e tre i generatori e attendendo qualche istante afficnhe il transitorio dovuto all'accensione svanisca, le misure si stabilizzano sui valori riportati nella tabella.
\subsection{Applicazione del teorema nel circuito}
Nell'ambito dell'esperimento possiamo applicare il teorema di Millman per ricavare la tensione di lato del resistore \(R_4\) (da 10 kohm).

In effetti possiamo considerare il collegamento più alto come un'unico nodo in cui si collegano tutti e quattro i resistori.
Utilizzando il teorema di MIllman possiamo ricavare il potenziale

\subsubsection{Verifica teorica}

\subsection{Stima della corrente di lato per un resistore}


Utilizzando la legge di Ohm e il valore della tensione misurato è possibile stimare la correnre che passa attravreso il resistore: \(I_4 = V_4 / R_4 = - 0.32 mA\).

Utilizzando le altre misure di corrente possiamo verificare che la stima calcolata sia giusta.

Infatti per la legge di Kirchhoff sui nodi deve essere che:
\[
\begin{aligned}
  &I_1 + I_2 + I_3 + I_4 = 0 \\
    &\Rightarrow  4.72 mA + 1.73 mA - 6.14 mA - 0.32 mA = -0.01
\end{aligned}
\]

Il risultato è abbastanza vicino allo zero da confermare la legge di Krichhoff e quindi da validare la corrente stimata sul resistore.

\section{Legge di Ohm}
\subsection{Descrizione esperimento}

Il circuito per questo esperimento consiste in un resistore \(R_1\) da 500 %\ohm \) attaccato ad un generatore di tensione \(V_s\).

L'esperimento consiste nel misurare ripetutamente la corrente che attraversa il resistore, cambiando ogni volta il potenziale ai capi dello stesso.

Le misure hanno lo scopo di verificare sperimentalmente la legge di Ohm.
I dati relatii alle misurazioni sono riportati in tabella.



Come si nota dal grafico l'intensità misurata lungo il circuito e la tensione fornita hanno una relazione di tipo lineare.

Con un modello di regressione lineare è possibile trovare una retta che rappresenti al meglio la relazione fra le due grandezze.

La retta che si ricava ha l'equazione \(I = 2.02 \frac{mA}{V} V_s - 0.14 mA\).

Il risulraro ottenuto coincide con la teoria, infatti la retta teorica calcolata a partire dalla legge di Ohm risulta \(I = \frac{V_s}{R} = 2 \frac{mA}{V} V_s\).

\subsection{Diagramma I/V}

\subsection{Relazione corrente-tensione}

\subsection{Stima della resistenza}
La prima legge di Ohm regola la caduta di potenziale lungo un resistore con l'intensità di corrente:
\(V=R \cdot I\).
Per stimare la resistenza del circuito possiamo utilizzare la retta generata dalla regressione lineare, infatti mettendola a sistema con la legge di Ohm possiamo ottenere una formula per \(R_1\).
\[
\begin{aligned}
\begin{cases}
    I=2.02 \frac{mA}{V} V_s -0.14 mA \\
    V_s= R_1 I
\end{cases}
\Rightarrow R_1 = \frac{V_s}{2.02 \frac{mA}{V}V_s - 0.14 mA}
\end{aligned}
\]


Utilizzando gli stessi valori \(V_s\) utilizzati per l'esperimento possiamo ricavare alcuni valori della resistenza \(R_1\).

A questo punto possiamo fare una media dei valori trovati e otteniamo che \(R_1 = 512 ohm\).

Lo stesso valore pote

\begin{table}[h!]
\centering
\begin{tabular}{|c|c|}
\hline
\textbf{Tensione (\( V \))} & \textbf{Corrente (\( I \))}  \\
\hline
0 V        & 0.180 \(\mu\) A   \\
0.5 V      & 0.7990 mA  \\
1 V        & 1.95547 mA \\
1.5 V      & 2.9315 mA  \\
2 V        & 3.9068 mA  \\
2.5 V      & 4.8840 mA  \\
3 V        & 5.8612 mA  \\
3.5 V      & 6.8366 mA  \\
4 V        & 7.8147 mA  \\
4.5 V      & 8.7913 mA  \\
5 V        & 9.7705 mA  \\
5.5 V      & 10.7454 mA \\
6 V        & 11.7247 mA \\
6.5 V      & 12.9857 mA \\
7 V        & 13.9810 mA \\
7.5 V      & 14.9860 mA \\
8 V        & 15.9883 mA \\
\hline
\end{tabular}
\caption{Dati di misura del diagramma \( I/V \) del resistore \( R_1 = 500 \, \Omega \)}
\end{table}


\section{Caratteristica \( I/V \) di una lampadina}
\subsection{Diagramma I/V della lampadina}
\subsection{Relazione corrente-tensione}

%Osservazioni rispetto punto 3
\subsection{Osservazioni}




\end{document}
