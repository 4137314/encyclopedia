% preamble.tex

\usepackage[utf8]{inputenc}
\usepackage[T1]{fontenc}
\usepackage[italian]{babel}
\usepackage{amsmath, amssymb, amsfonts}
\usepackage{mathtools}
\usepackage{physics}
\usepackage{bm}
\usepackage{geometry}
\usepackage{hyperref}
\usepackage{fancyhdr}
\usepackage{graphicx}
\usepackage{xcolor}
\usepackage{enumitem}
\usepackage{caption}

\geometry{a4paper, margin=2.5cm}

\hypersetup{
  colorlinks=true,
  linkcolor=blue!70!black,
  citecolor=green!50!black,
  urlcolor=cyan!60!black,
  pdftitle={Engineering Electromagnetics},
  pdfauthor={Mattia Fait}
}

\pagestyle{fancy}
\fancyhf{}
\lhead{\textit{Engineering Electromagnetics}}
\rhead{\thepage}

\newtheorem{theorem}{Teorema}[section]
\newtheorem{definition}{Definizione}[section]
\newtheorem{lemma}{Lemma}[section]
\newtheorem{proposition}{Proposizione}[section]
\newtheorem{corollary}{Corollario}[section]

\newcommand{\R}{\mathbb{R}}
\newcommand{\N}{\mathbb{N}}
\newcommand{\Z}{\mathbb{Z}}
\newcommand{\Q}{\mathbb{Q}}
\newcommand{\C}{\mathbb{C}}
\newcommand{\eps}{\varepsilon}

