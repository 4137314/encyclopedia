\documentclass[a4paper,12pt]{article}

% Pacchetti per la matematica e simboli
\usepackage{amsmath}    % Formule matematiche
\usepackage{amssymb}    % Simboli matematici
\usepackage{amsfonts}   % Font matematici
\usepackage{mathrsfs}   % Scrittura calligrafica matematica
\usepackage{amsthm}     % Teoremi e definizioni

% Pacchetti per grafici
\usepackage{graphicx}   % Inclusione grafica
\usepackage{tikz}       % Disegni e grafici
\usepackage{pgfplots}   % Grafici avanzati
\pgfplotsset{compat=1.18} % Imposta compatibilità per pgfplots

% Pacchetti per la gestione degli indici e riferimenti
\usepackage{hyperref}   % Riferimenti ipertestuali
\hypersetup{
    colorlinks=true,    % Colori per i link
    linkcolor=darkblue, % Colore dei link interni
    citecolor=darkblue, % Colore dei riferimenti alle citazioni
    filecolor=darkblue, % Colore dei link ai file
    urlcolor=darkblue,  % Colore degli URL
    pdftitle={Algebra elementare},
    pdfpagemode=UseOutlines
}
\usepackage{tocbibind}  % Include l'indice e la bibliografia nel sommario
\usepackage{fancyhdr}   % Intestazioni e piè di pagina personalizzati
\usepackage{bookmark}   % Gestione segnalibri PDF

% Configurazione per intestazioni e piè di pagina
\setlength{\headheight}{15pt} % Altezza dell'intestazione
\addtolength{\topmargin}{-10pt} % Riduzione del margine superiore

\pagestyle{fancy}
\fancyhf{}  % Pulisce intestazione e piè di pagina
\fancyhead[L]{Algebra elementare}  % Intestazione sinistra
\fancyhead[R]{\today}                      % Intestazione destra
\fancyfoot[C]{\thepage}                    % Numero di pagina al centro

% Impostazioni dei margini
\usepackage{geometry}
\geometry{
    left=1.2in,
    right=1.2in,
    top=1in,
    bottom=1.2in
}

% Gestione delle sezioni e sottosezioni
\usepackage{titlesec}
\titleformat{\section}[block]{\large\scshape}{\thesection}{1em}{} % Titoli sezioni maiuscoli
\titleformat{\subsection}[block]{\normalsize\bfseries}{\thesubsection}{1em}{}
\titleformat{\subsubsection}[block]{\normalsize\itshape}{\thesubsubsection}{1em}{}
\titleformat{\paragraph}[runin]{\normalsize\bfseries}{\theparagraph}{1em}{}[:]
\titleformat{\subparagraph}[runin]{\normalsize\itshape}{\thesubparagraph}{1em}{}[:]

% Colore e aspetto
\usepackage{xcolor}
\definecolor{darkblue}{rgb}{0.0, 0.2, 0.6}
\definecolor{gray}{rgb}{0.5, 0.5, 0.5}

% Teoremi e definizioni formali
\newtheoremstyle{mystyle} % Definisci lo stile del teorema
  {10pt} % Spaziatura sopra
  {10pt} % Spaziatura sotto
  {\itshape} % Corpo del teorema in corsivo
  {} % Indentazione del numero
  {\bfseries} % Font del titolo
  {}     % Punteggiatura dopo il titolo
  {\newline} % Spazio dopo il titolo
  {}     % Stile del testo

\theoremstyle{mystyle}
\newtheorem{theorem}{Teorema}[section]
\newtheorem{definition}[theorem]{Definizione}
\newtheorem{lemma}[theorem]{Lemma}
\newtheorem{corollary}[theorem]{Corollario}
\newtheorem{proposition}[theorem]{Proposizione}

% Pacchetti aggiuntivi
\usepackage{enumitem}   % Gestione elenchi personalizzati
\usepackage{multicol}   % Colonne multiple
\usepackage{booktabs}   % Tabelle formattate professionalmente
\usepackage{caption}    % Migliore gestione delle didascalie
\captionsetup{
    labelfont=bf,
    font=small,
    labelsep=colon
}

% Impostazioni di line spacing
\usepackage{setspace}
\onehalfspacing  % Interlinea 1.5 per migliorare la leggibilità

% Informazioni del documento
\title{\textbf{Elettromagnetismo}}
\author{\textit{Oudeys}}
\date{\today}

% Inizio del documento
\begin{document}

\maketitle

\tableofcontents
\newpage


\section{Elettrostatica}
\subsection{Polarizzazione dei dielettrici}
\subsubsection{Costante dielettrica del vuoto}
\[\epsilon_0 = 8.85 \cdot 10^{-12} \frac{C^2}{N \cdot m^2}\]
\subsubsection{Costante dielettrica relativa}
\[\epsilon_r = \frac{\epsilon}{\epsilon_0}\]
\subsubsection{Costante dielettrica assoluta}
\[\epsilon = \epsilon_r \cdot \epsilon_0\]

\newpage

\subsection{Carica elettrica e legge di Coulumb}
\subsubsection{Carica elettrica}
\([Q]=[T][i]\)\\
\(C\)
\subsubsection{Legge di Coulumb}
\[F = K \cdot \frac{Q_1 \cdot Q_2}{r^2}\]
\paragraph{Costante di Coulumb}
\(K = \frac{1}{4 \pi \epsilon} = \frac{1}{4\pi \epsilon_0 \epsilon_r}\)
\subsubsection{Legge di Coulumb nel vuoto}
\[F = K_0 \cdot \frac{Q_1 \cdot Q_2}{r^2}\]
\paragraph{Costante di Coulumb nel vuoto}
\(K_0 = \frac{1}{4 \pi \epsilon_0} = 9 \cdot 10^9 \frac{N \cdot m^2}{C^2}\)

\newpage


\subsection{Campo elettrico}
\subsubsection{Intensità del campo elettrico}
\([E]=[L][M][T]^{-3}[i]^{-1}\)\\
\(\frac{N}{C}=\frac{V}{m}\)\\

\[
\begin{aligned}
    E &= \frac{F}{q} \\
    &= K \cdot \frac{Q \cdot q}{r^2} \cdot \frac{1}{q}\\
    &= K \cdot \frac{Q}{r^2}
\end{aligned}\]\\
\[\textbf{E} = \lim_{q \rightarrow 0} \frac{\textbf{F}}{q}\]
\subsubsection{Principio di sovrapposizione}
\[
\begin{aligned}
    \textbf{E} &= \textbf{E}_1 + \textbf{E}_2 + \ldots + \textbf{E}_n \\
    &= \sum_i \textbf{E}_i
\end{aligned}\]
\subsubsection{Campo generato da un dipolo elettrico}
\[E_r = k \cdot \frac{2 p \cdot \cos \theta}{r^3}\]\\
\[E_\theta = k \cdot \frac{2p \cdot \sin \theta}{r^3}\]
\paragraph{Momento di dipolo}
\[\vec p = q \cdot \vec a\]

\newpage


\subsection{Energia potenziale elettrica}
\([E_p]=[L]^2[M][T]^{-2}\)\\
\(J\)\\
\[E_p = K \cdot \frac{Q \cdot q}{r}\]

\newpage


\subsection{Potenziale elettrico}
\([V_p]=[L]^2[M][T]^{-3}[i]^{-1}\)\\
\(V= \frac{J}{C}\)\\
\[\begin{aligned}
    V_p &= \frac{E_p}{q} \\
    &= K \cdot \frac{Q \cdot q}{r} \cdot \frac{1}{q}\\
    &= K \cdot \frac{Q}{r}
\end{aligned}\]
\subsubsection{Potenziale elettrico per un sistema di cariche}
\[
\begin{aligned}
    V & = V_1 + V_2 + \ldots + V_n\\
    &= K \cdot \left( \frac{Q_1}{r_1} + \frac{Q_2}{r_2} + \ldots + \frac{Q_n}{r_n}\right ) \\
    &= \sum_i V_i\\
    &= \sum_i \frac{Q_i}{r_i}
\end{aligned}\]
\subsubsection{Potenziale generato da un dipolo elettrico}
\[V = K \cdot \frac{p \cdot \cos \theta}{r^2}\]\\
\(\theta < \frac{\pi}{2} \Rightarrow V >0\)\\
\(\theta > \frac{\pi}{2} \Rightarrow V < 0\)
\subsubsection{Tensione - Differenza di potenziale elettrico}
\[V_A - V_B\]\\
\[
\begin{aligned}
    L_{a \rightarrow B}& = E_A - E_B \\
    &= - \Delta E \\
    &= q \cdot (V_A - V_B)\\
    &= -q \cdot \Delta V
\end{aligned}\]
\subsubsection{Superfici equipotenziali}
\[L = q \cdot (V_A - V_B) = 0\]


\newpage


\subsection{Teorema di Gauss}
\subsubsection{Flusso di un vettore}
\[\Phi_S (\textbf{E}) = \textbf{E} \cdot \textbf{S} = E \cdot S \cdot \cos \theta = E_n \cdot S\]
\subsubsection{Teorema di Gauss}
\[\Phi_S (\textbf{E}) = \frac{Q}{\epsilon _0}\]\\
\[\Phi _S (\textbf{E}) = \frac{1}{\epsilon_0} \cdot \sum_i Q_i\]
\subsubsection{Conduttore sferico}
\[E = K \cdot \frac{Q}{l^2}\]\\
\[V = K \cdot \frac{Q}{l}\]
\subsubsection{Campo elettrico generato da un piano infinito di carica}
\[E = \frac{\sigma}{2 \epsilon_0}\]
\paragraph{Densità superficiale di carica elettrica}
\[\sigma = \frac{\Delta Q}{\Delta S}\]

\newpage

 
\subsection{Capacità di una conduttore}
\subsubsection{Capacità elettrostatica}
\([C]=[L]^{-2}[M]^{-1}[T]^4[i]^2\)\\
\[C = \frac{Q}{V}\]\\
\[F= \frac{C}{V}\]
\subsubsection{Capacità di un conduttore sferico}
\[C = \frac{Q}{V} = \frac{Q \cdot R}{K \cdot Q} = \frac{R}{K} = 4 \pi \epsilon_0 \cdot R\]


\newpage


\subsection{Condensatori}
\subsubsection{Capacità di un condensatore}
\[C = \frac{Q}{\Delta V}\]
\subsubsection{Condensatore piano}
\[C = \epsilon \cdot \frac{S}{d}\]
\subsubsection{Condensatore sferico}
\[C = \frac{4 \pi \epsilon \cdot R_1 \cdot R_2}{R_2 - R_1}\]
\subsubsection{Condensatore cilindrico}
\[C = \frac{2 \pi \epsilon \cdot l}{\log \left(\frac{R_2}{R_1} \right )}\]
\subsubsection{Effetto di un dielettrico in un condensatore}
\[\frac{C}{C_0} = \epsilon_r\]
\subsubsection{Lavoro di carica di un condensatore}
\[L = \frac{1}{2} \cdot \frac{Q^2}{C} = \frac{1}{2} \cdot C \cdot \Delta V^2\]
\subsubsection{Condensatori in parallelo}
\[C_1 = \frac{Q_1}{\Delta V } \ldots C_n = \frac{Q_n}{\Delta V}\]
\paragraph{Capacità equivalente}
\[
{C_{eq}= C_1+C_2+ \ldots + C_n = \frac{Q_1+Q_2+\ldots+Q_n}{\Delta V} = \frac{Q}{\Delta V}}
\]
\paragraph{Condensatori in serie}
\[\frac{1}{C_{eq}} = \frac{1}{C_1}+\frac{1}{C_2} + \ldots + \frac{1}{C_n}\]
\paragraph{Elettronvolt}
\[L =q \cdot \Delta V\]

\newpage


\section{Corrente elettrica}
\subsection{Intensità di corrente}
\([i]\)\\
\(A= \frac{C}{s}\)\\
\[i = \frac{\Delta Q}{\Delta t}\]

\subsubsection{Effetto Volta}
\[L = e \cdot V_i\]
\subsubsection{Forza elettromotrice}
\([fem]=[L]^2[M][T]^{-3}[i]^{-1}\)\\
\(V= \frac{J}{C}\)\\
\[fem = \frac{\Delta L}{\Delta q}\]

\newpage


\subsection{Le leggi di Ohm}
\subsubsection{I legge di Ohm}
\[V_B-V_A = \Delta V = R \cdot i \Leftrightarrow R = \frac{\Delta V}{i}\]
\paragraph{Resistenza}
\[R = R_0 \cdot(1 + \alpha \cdot t)\]
\paragraph{Conduttanza}
\[c = \frac{1}{R}\]
\subsubsection{II legge di Ohm}
\[R = \rho \cdot \frac{l}{S}\]
\paragraph{Conducibilità del materiale}
\[\lambda = \frac{1}{\rho}\]
\subparagraph{Resistività dei metalli}
\[\rho = \rho_0 \cdot (1 + \alpha \cdot t)\]

\newpage


\subsection{Effetto Joule}
\subsubsection{Energia dissipata}
\[L = R \cdot i ^2 \cdot \Delta t = \Delta V \cdot i \cdot \Delta t = \frac{(\Delta V)^2}{R} \cdot \Delta t\]
\subsubsection{Potenza dissipata}
\[P = \frac{L}{\Delta t} = R \cdot i^2 \cdot = \Delta V \cdot i = \frac{(\Delta V)^2}{R}\]
\subsubsection{Calore dissipato}
\[ Q = \frac{R \cdot i^2 \cdot \Delta t}{4,186  J/cal}\]

\newpage


\subsection{Resistenze in serio e in parallelo}
\subsubsection{Resistenze in serie}
\[R_{eq}= R_1 + R_2 + \ldots + R_n = \sum_i R_i\]
\subsubsection{Resistenze in parallelo}
\[\frac{1}{R_{eq}} = \frac{1}{R_1}+ \frac{1}{R_2}+ \ldots + \frac{1}{R_n} = \sum_i \frac{1}{R_i}\]
\subsubsection{Legge di Ohm per i circuiti chiusi}
\[fem = (R+r) \cdot i = \Delta V + r \cdot i\]

\newpage


\subsection{Leggi di Kirchoff}
\subsubsection{I legge di Kirchoff}
\[\sum_i i_i = 0\]
\subsubsection{II legge di Kirchoff}
\[\sum_i fem _i = \sum_i i_i \cdot R_i\]

\newpage


\subsection{Corrente elettrica nei liquidi}
\subsubsection{Leggi di Faraday}
\paragraph{I legge di Faraday}
\[M = k \cdot Q\]
\paragraph{II legge di Faraday}

\newpage


\subsection{Corrente elettrica nei gas}
\subsubsection{Legge di Paschen}
\[V = k \cdot P \cdot d\]


\newpage


\section{Magnetismo}
\subsection{Campo magnetico}
\newpage

\subsection{Campo magnetico generato da una corrente elettrica}
\newpage

\subsection{Induzione elettromagnetica}
\newpage

\subsection{Correnti alternate}
\newpage

\subsection{Elettromagnetismo}

\end{document}
