\documentclass[a4paper,12pt]{article}

% Pacchetti per la matematica e simboli
\usepackage{amsmath}    % Formule matematiche
\usepackage{amssymb}    % Simboli matematici
\usepackage{amsfonts}   % Font matematici
\usepackage{mathrsfs}   % Scrittura calligrafica matematica
\usepackage{amsthm}     % Teoremi e definizioni

% Pacchetti per grafici
\usepackage{graphicx}   % Inclusione grafica
\usepackage{tikz}       % Disegni e grafici
\usepackage{pgfplots}   % Grafici avanzati
\pgfplotsset{compat=1.18} % Imposta compatibilità per pgfplots

% Pacchetti per la gestione degli indici e riferimenti
\usepackage{hyperref}   % Riferimenti ipertestuali
\hypersetup{
    colorlinks=true,    % Colori per i link
    linkcolor=darkblue, % Colore dei link interni
    citecolor=darkblue, % Colore dei riferimenti alle citazioni
    filecolor=darkblue, % Colore dei link ai file
    urlcolor=darkblue,  % Colore degli URL
    pdftitle={Algebra elementare},
    pdfpagemode=UseOutlines
}
\usepackage{tocbibind}  % Include l'indice e la bibliografia nel sommario
\usepackage{fancyhdr}   % Intestazioni e piè di pagina personalizzati
\usepackage{bookmark}   % Gestione segnalibri PDF

% Configurazione per intestazioni e piè di pagina
\setlength{\headheight}{15pt} % Altezza dell'intestazione
\addtolength{\topmargin}{-10pt} % Riduzione del margine superiore

\pagestyle{fancy}
\fancyhf{}  % Pulisce intestazione e piè di pagina
\fancyhead[L]{Meccanica Newtoniana}  % Intestazione sinistra
\fancyhead[R]{\today}                      % Intestazione destra
\fancyfoot[C]{\thepage}                    % Numero di pagina al centro

% Impostazioni dei margini
\usepackage{geometry}
\geometry{
    left=1.2in,
    right=1.2in,
    top=1in,
    bottom=1.2in
}

% Gestione delle sezioni e sottosezioni
\usepackage{titlesec}
\titleformat{\section}[block]{\large\scshape}{\thesection}{1em}{} % Titoli sezioni maiuscoli
\titleformat{\subsection}[block]{\normalsize\bfseries}{\thesubsection}{1em}{}
\titleformat{\subsubsection}[block]{\normalsize\itshape}{\thesubsubsection}{1em}{}
\titleformat{\paragraph}[runin]{\normalsize\bfseries}{\theparagraph}{1em}{}[:]
\titleformat{\subparagraph}[runin]{\normalsize\itshape}{\thesubparagraph}{1em}{}[:]

% Colore e aspetto
\usepackage{xcolor}
\definecolor{darkblue}{rgb}{0.0, 0.2, 0.6}
\definecolor{gray}{rgb}{0.5, 0.5, 0.5}

% Teoremi e definizioni formali
\newtheoremstyle{mystyle} % Definisci lo stile del teorema
  {10pt} % Spaziatura sopra
  {10pt} % Spaziatura sotto
  {\itshape} % Corpo del teorema in corsivo
  {} % Indentazione del numero
  {\bfseries} % Font del titolo
  {}     % Punteggiatura dopo il titolo
  {\newline} % Spazio dopo il titolo
  {}     % Stile del testo

\theoremstyle{mystyle}
\newtheorem{theorem}{Teorema}[section]
\newtheorem{definition}[theorem]{Definizione}
\newtheorem{lemma}[theorem]{Lemma}
\newtheorem{corollary}[theorem]{Corollario}
\newtheorem{proposition}[theorem]{Proposizione}

% Pacchetti aggiuntivi
\usepackage{enumitem}   % Gestione elenchi personalizzati
\usepackage{multicol}   % Colonne multiple
\usepackage{booktabs}   % Tabelle formattate professionalmente
\usepackage{caption}    % Migliore gestione delle didascalie
\captionsetup{
    labelfont=bf,
    font=small,
    labelsep=colon
}

% Impostazioni di line spacing
\usepackage{setspace}
\onehalfspacing  % Interlinea 1.5 per migliorare la leggibilità

% Informazioni del documento
\title{\textbf{Meccanica Newtoniana}}
\author{\textit{Oudeys}}
\date{\today}

% Inizio del documento
\begin{document}

\maketitle


\tableofcontents
\newpage


\section{Cinematica}
\subsection{Moto in una dimensione}
\subsubsection{Velocità vettoriale media}
\([v]=[L][T]^{-1}\)\\
\(m \cdot s^{-1}\)\\
\[\overline v = \frac{x_2-x_1}{t_2-t_1} = \frac{\Delta x}{\Delta t}\]\\
\(\frac{km}{h}=  \frac{1}{3.6} \cdot \frac{m}{s}\)\\
\( \frac{m}{s} = 3.6 \cdot  \frac{km}{h}\)
\subsubsection{Velocità vettoriale istantanea}
\[v = \lim_{\Delta \rightarrow 0} \frac{\Delta x}{\Delta t} = \frac{dx}{dt}\]
\subsubsection{Accelerazione vettoriale media}
\([a]=[L][T]^{-2}\)\\
\(m \cdot s^{-2}\)\\
\[\overline a = \frac{v_2-v_1}{t_2-t_1} = \frac{\Delta v}{\Delta t}\]



\subsubsection{Accelerazione vettoriale istantanea}
\[\begin{aligned}
    a &= \lim_{\Delta t \rightarrow 0} \frac{\Delta v}{\Delta t} \\
    &= \frac{dv}{dt}\\
    &= \frac{d^2x}{dt^2}
\end{aligned}\]
\subsubsection{Moto uniformemente accelerato}

\begin{table}[h]
    \centering
    \begin{tabular}{|c|}
    \hline
    \(\begin{array}{c}
        v = v_0 + at \\

        x = x_0 + v_0 t + \frac{1}{2} a t^2 \\

        v^2 = v_0^2 + 2a(x - x_0) \\
        \overline{v} = \frac{v + v_0}{2}
    \end{array}\) \\
    \hline
    \end{tabular}
\end{table}



\newpage

\subsection{Moto in più dimensioni}
\subsubsection{Vettore spostamento}
\[\Delta \vec r =(x_2-x_1)\hat i + (y_2-y_1)\hat j + (z_2-z_1)\hat k\]
\subsubsection{Velocità vettoriale media}
\[\overline v = \frac{\Delta \vec r}{\Delta t}\]
\subsubsection{Velocità vettoriale istantanea}
\[
\begin{aligned}
    \vec v &= \lim_{\Delta t \rightarrow 0} \frac{\Delta \vec r}{\Delta t}\\
    &= \frac{d \vec r}{dt} \\
    &= \frac{dx}{dt} \hat i +\frac{dy}{dt} \hat j+ \frac{dz}{dt} \hat k \\
    &= v_x \hat i + v_y \hat j + v_z \hat k
\end{aligned}
\]
\subsubsection{Accelerazione vettoriale media}
\[\frac{\Delta \vec v}{\Delta t} = \frac{\vec v_2 - \vec v_1}{t_2-t_1}\]
\subsubsection{Accelerazione vettoriale istantanea}
\[\begin{aligned}
    \vec a &= \lim_{\Delta t \rightarrow 0} \frac{\Delta \vec v}{\Delta t} \\
    &= \frac{d \vec v}{dt} \\
    &= \frac{dv_x}{dt}\hat i +\frac{dv_y}{dt}\hat j +\frac{dv_z}{dt}\hat k\\
    &= \frac{d^2x}{dt^2}\hat i +\frac{d^2y}{dt^2}\hat j +\frac{d^2z}{dt^2}\hat k
\end{aligned}\]
\subsubsection{Moto uniformemente accelerato}
\[\vec r = \vec r_0 + \vec v_0 t + \frac{1}{2} \vec a t^2\]\\

\begin{table}[h]
    \centering
    \begin{tabular}{|l|l|}
    \hline
    \textbf{Componente x} & \textbf{Componente y} \\
    \hline
    \( v_x = v_{x0} + a_x t \) & \( v_y = v_{y0} + a_y t \) \\
    \( x = x_0 + v_{x0} t + \frac{1}{2} a_x t^2 \) & \( y = y_0 + v_{y0} t + \frac{1}{2} a_y t^2 \) \\
    \( v_x^2 = v_{x0}^2 + 2a_x (x - x_0) \) & \( v_y^2 = v_{y0}^2 + 2a_y (y - y_0) \) \\
    \hline
    \end{tabular}
\end{table}

\subsubsection{Moto parabolico}
\[y= \left( \frac{v_{y0}}{v_{x0}}\right)x-\left( \frac{g}{2v_{x0}^2}\right)x^2\]\\

\begin{table}[h]
    \centering
    \begin{tabular}{|l|l|}
    \hline
    \textbf{Componente x } & \textbf{Componente y } \\
    \hline 
    \(\forall \quad a_x=0\) & \(\forall \quad a_y=-g\) \\
    \( v_x = v_{x0} \) & \( v_y = v_{y0} - gt \) \\
    \( x = x_0 + v_{x0} t \) & \( y = y_0 + v_{y0} t - \frac{1}{2} g t^2 \) \\
    \( v_y^2 = v_{y0}^2 + 2g (y - y_0) \) & \\
    \hline
    \end{tabular}
\end{table}

\paragraph{Gittata}
\[
\begin{aligned}
    R &= v_{x0}t \\
    &= v_{x0} \left(\frac{2v_{y0}}{g}\right) \\
    &= \frac{2v_{x0}v_{y0}}{g} \\
    &= \frac{2v_0^2 \sin \theta_0 \cos \theta_0}{g} \\
    &= \frac{v_0^2 \sin 2 \theta_0}{g}
\end{aligned}
\]
\\
\(R_{max} = \frac{v_0^2}{g} \Leftrightarrow \theta = 45^\circ\)

\newpage

\subsection{Moto rotazionale}
\subsubsection{Velocità angolare media}
\[\overline \omega = \frac{\Delta \theta}{\Delta t}\]
\subsubsection{Velocità angolare istantanea}
\[\begin{aligned}
    \omega &= \lim_{\Delta t \rightarrow 0} \frac{\Delta \theta}{\Delta t}\\
    &= \frac{d \theta}{dt}
\end{aligned}\]
\subsubsection{Accelerazione angolare media}
\[\overline \alpha = \frac{\Delta \omega}{\Delta t}\]
\subsubsection{Accelerazione angolare istantanea}
\[\begin{aligned}
    \alpha &= \lim_{\Delta t \rightarrow 0} \frac{\Delta \omega}{\Delta t}\\
    &= \frac{d \omega}{dt}
\end{aligned}\]
\subsubsection{Frequenza}
\[f=\frac{\omega}{2\pi}\]
\subsubsection{Periodo}
\[T = \frac{1}{f}\]

\subsubsection{Equazioni del moto in caso di accelerazione angolare costante}
\begin{table}[h]
    \centering
    \begin{tabular}{|c|}
    \hline
    $\begin{array}{c}
        \omega = \omega_0 + \alpha t \\
        \theta = \omega_0 t + \frac{1}{2} \alpha t^2 \\
        \omega^2 = \omega_0^2 + 2 \alpha \theta \\
        \overline{\omega} = \frac{\omega + \omega_0}{2}
    \end{array}$\\
    \hline
    \end{tabular}
\end{table}



\newpage


\section{Dinamica}

\subsection{Leggi di Newton}
\subsubsection{I principio della dinamica}
\[\vec F = 0 \Rightarrow \vec a = 0 \Rightarrow \vec v = \text{costante}\]
\subsubsection{II principio della dinamica}
\([F]=[M][L][T]^{-2}\)\\
\(N=kg \cdot m \cdot s^{-2}\)\\
\[\sum \vec F = m \vec a\]
\subsubsection{III principio della dinamica}
\[\vec F_{AB} = - \vec F_{BA}\]
\subsubsection{Definizione di massa}
\(\forall F_1=F_2\)\\
\[\frac{m_2}{m_1}=\frac{a_1}{a_2}\]

\newpage

\subsection{Moto circolare uniformemente accelerato}
\subsubsection{Accelerazione centripeta radiale}
\[a_R = \frac{v^2}{r}\]
\subsubsection{Forza centripeta radiale}
\[\begin{aligned}
    \sum F_R &= m \cdot a_R\\
    &= m \cdot \frac{v^2}{r}
\end{aligned}\]
\subsubsection{Forza normale su curva inclinata}
\[F_N = m \cdot \frac{v^2}{r \cdot \sin \theta}\]







\newpage


\subsection{Gravitazione}
\subsubsection{Legge di Newton della gravitazione universale}
\[F_G = G \cdot \frac{m_1\cdot m_2}{r^2}\]\\
\[g= G \cdot\frac{M_T}{r_T^2}\]\\
\[M_T=\frac{g \cdot r_T^2}{G}\]
\paragraph{Costante di gravitazione universale}
\(G= 6.67 \cdot 10^{-11} \cdot \frac{N \cdot m^2}{kg^2}\)
\subsubsection{Leggi di Keplero}
\paragraph{I legge di Keplero}\mbox{}\\
\parbox{0.8\textwidth}{
\text{La traiettoria di ogni pianeta attorno al Sole è un'ellisse, con il Sole che occupa uno dei due fuochi.}
}

\paragraph{II legge di Keplero}\mbox{}\\
\parbox{0.8\textwidth}{
Ogni pianeta si muove in modo che la proiezione di una linea immaginaria tracciata dal Sole al pianeta disegni aree uguali in tempi uguali.
}

\paragraph{III legge di Keplero}\mbox{}\\
\parbox{0.8\textwidth}{
Il rapporto tra i quadrati dei periodi di due pianeti che orbitano intorno al Sole è pari al rapporto dei cubi dei loro semiassi maggiori. 
Il semiasse maggiore è la metà della lunghezza maggiore dell'asse dell'orbita e rappresenta la distanza media del pianeta dal Sole.
} \\
\[\frac{a_1^3}{T_1^2} = \frac{a_2^3}{T_2^2}\]

\subsubsection{Campo gravitazionale}
\[\vec g = \frac{\vec F}{m}\]
\subsubsection{Principio di equivalenza}
\parbox{0.8\textwidth}{
Non esiste alcun esperimento che possa distinguere se l'accelerazione di un corpo sia causata dalla forza di gravità o dal fatto che è il sistema di riferimento che sta accelerando.
}

\newpage

\subsection{Attrito}
\subsubsection{Attrito dinamico}
\[F= \mu _d F_N\]
\subsubsection{Attrito statico}
\[F < \mu _s F_N\]
\subsubsection{Attrito viscoso}
\[F_V =-bv\]
\paragraph{Velocità limite caduta libera}
\[v_L = \frac{mg}{b}\]

\newpage



\subsection{Lavoro ed energia}
\subsubsection{Lavoro di una forza costante}
\([L]=[M][L]^2[T]^{-2}\)\\
\(N \cdot m = J\)\\
\[W = F \cdot d \cdot \cos \theta\]

\subsubsection{Lavoro di una forza variabile}
\[
W = \begin{aligned}[t]
    &\phantom{=} \lim_{\Delta l_i \rightarrow 0} \sum F_i \cdot \cos \theta_i \Delta l_i \\
    &= \int_A^B F \cdot \cos \theta \cdot dl \\
    &= \int_A^B \vec{F} \cdot d\vec{l} \\
    &= \int_{x_A}^{x_B} F_x \, dx + \int_{y_A}^{y_B} F_y \, dy + \int_{z_A}^{z_B} F_z \, dz
\end{aligned}
\]

\paragraph{Lavoro compiuto da una forza elastica}

\subsubsection{Energia cinetica}
\[K = \frac{1}{2}mv^2\]
\subsubsection{Teorema dell'energia cinetica}
\[\begin{aligned}
    W_{tot} &= \Delta K\\
    &= \frac{1}{2} m v_2^2- \frac{1}{2}mv_1^2
\end{aligned}\]\\
\parbox{0.8\textwidth}{
Il lavoro totale compiuto su un corpo è uguale alla variazione dell'energia cinetica del corpo.
Il lavoro compiuto da una forza conservativa è recuperabile.
}

\newpage


\subsection{Conservazione dell'energia}
\subsubsection{Forze conservative e non conservative}
\parbox{0.8\textwidth}{
Una forza è conservativa se il lavoro compiuto dalla forza lungo un qualunque percorso chiuso è zero.
}
\\

\begin{table}[h]
    \centering
    \begin{tabular}{|c|c|}
        \hline
        \textbf{Forze conservative} & \textbf{Forze non conservative} \\
        \hline
        Forza gravitazionale & Forza di attrito \\
        Forza elastica & Tensione \\
        Forza elettrica & Forza di propulsione \\
        \hline
    \end{tabular}
\end{table}


\paragraph{Lavoro della forza di gravità}
\[\begin{aligned}
    W_G &= \int_1^2 \vec F_G \cdot d \vec l\\
    &= \int_1^2 mg \cdot \cos \theta dl \\
    &=- \int_1^2 mg dy = -mg(y_2-y_1)
\end{aligned}\]

\subsubsection{Energia potenziale}
\paragraph{Energia potenziale gravitazionale}
\[\begin{aligned}
    \Delta U & = U_2 - U_1\\
    &= - W_G \\
    &= mg (y_2-y_1)
\end{aligned}\]\\
\[U_G= mgy\]
\paragraph{Variazione di energia potenziale}
\[\begin{aligned}
    \Delta U &= U_2 - U_1\\
    &= - \int_1^2 \vec F \cdot d \vec l\\
    &= - W
\end{aligned}\]
\paragraph{Energia potenziale elastica}
\[\begin{aligned}
\Delta U &= U(x) - U(0) \\
&= - \int _1^2 \vec F \cdot d \vec l\\
&= - \int_0^x(-kx)dx \\
&= \frac{1}{2}kx^2
\end{aligned}\]
\paragraph{Energia potenziale di una forza unidimensionale}
\[U(x)= - \int F(x) dx\]\\
\[F(x)= - \frac{d U(x)}{dx}\]
\paragraph{Energia potenziale in tre dimensioni}
\[\vec F(x,y,z) = - \hat i \frac{\partial U}{\partial x} - \hat j \frac{\partial U}{\partial y} - \hat k \frac{\partial U}{\partial z}\]
\subsubsection{Energia meccanica e sua conservazione}
\paragraph{Energia meccanica}
\[E=K+U\]
\paragraph{Conservazione dell'energia meccanica}
\[E = \frac{1}{2}mv^2 + U \]\\
\[\frac{1}{2}m v_1^2 + mgy_1 = \frac{1}{2}mv_2^2+mgy_2\]\\
\[\frac{1}{2} mv_1^2+ \frac{1}{2} k x_1^2 = \frac{1}{2}v_2^2 + \frac{1}{2}k x_2^2\]
\subsubsection{Principio di conservazione dell'energia}
\[\Delta K + \Delta U = 0\]
\subsubsection{Energia potenziale gravitazionale e velocità di fuga}
\[\vec F = - G \frac{mM_T}{r^2 \hat r}\]\\
\[\Delta U = - \frac{GmM_T}{r_2}+ \frac{GmM_T}{r_1} \]\\
\[U(r) = - \frac{GmM_T}{r}\]\\
\[\frac{1}{2}mv_1^2 - G \frac{mM_T}{r_1}= \frac{1}{2}mv_2^2- G \frac{mM_T}{r_2}\]
\paragraph{Velocità di fuga}
\[v_f = \sqrt{2 G M_t / r_T} = 1.12 \time 10^4 m/s\]

\subsection{Potenza}
\subsubsection{Potenza media}
\([P]=[M][L]^2[T]^{-3}\)\\
\(W=\frac{J}{s}\)\\
\[\overline P = \frac{W}{t}\]

\subsubsection{Potenza istantanea}
\[P = \frac{dW}{dt} = \frac{dE}{dt} = \vec F \cdot \frac{d \vec l}{dt} = \vec F \cdot \vec v\]
\subsubsection{Efficienza}
\[e = \frac{P_{out}}{P_{in}}\]

\newpage


\subsection{Quantità di moto}
\subsubsection{Quantità di moto e sua relazione con la forza}
\([P]=[M][L][T]^{-1}\)\\
\(kg \cdot m \cdot s^{-1}\)\\
\[\vec p = m \vec v\]
\paragraph{II principio della dinamica}
\[
\begin{aligned}
    \sum \vec F &= \frac{d \vec p}{dt}\\
    &= \frac{d(m \vec v)}{dt}\\
    &= m \frac{d \vec v}{dt} \\
    &= m \vec a
\end{aligned}\]

\subsubsection{Conservazione della quantità di moto}
\(\forall \sum \vec F_{ext} = 0\)\\
\[m_a \vec v_a + m_B \vec v_B = m_a \vec v'_a + m_B \vec v'_B\]
\paragraph{II principio della dinamica per un sistema di corpi}
\[\frac{d \vec P}{dt} = \sum \vec F_{ext}\]
\subsubsection{Urti e impulso}
\[d \vec p = \vec F dt\]
\paragraph{Impulso}
\[
\begin{aligned}
    \vec J &= \Delta \vec p \\
    &= \vec p_f- \vec p_i \\
    &= \int_{t_i}^{t_f} \vec F dt
\end{aligned}\]

\subsubsection{Conservazione dell'energia e della quantità di moto negli urti}
\paragraph{Urti elastici in una dimensione}
\[\frac{1}{2} m_Av_A^2 + \frac{1}{2} m_B v_B^2 = \frac{1}{2} m_Av'_A^2+\frac{1}{2} m_Bv'_B^2\]
\paragraph{Urti anaelastici}
\paragraph{Urti in due o tre dimensioni}
\subsubsection{Centro di massa}
\[\vec r_{CM}= \frac{\sum m_i \vec r_i}{M}\]
\subsubsection{Centro di massa e moto traslatorio}

\newpage




\subsection{Momento angolare}

\subsubsection{II principio della dinamica per il moto rotazionale}
\[\sum \tau = \frac{dL}{dt}\]

\subsubsection{Conservazione del momento angolare}
\subsubsection{Momento angolare di una particella}
\subsubsection{Momento angolare e momento delle forze per un sistema di particelle}
\subsubsection{Momento angolare e momento delle forze per un corpo rigido}
\subsubsection{Conservazione del momento angolare}
\subsubsection{Giroscopio}
\subsubsection{Forze d'inerzia}
\subsubsection{Effetto Coriolis}

\newpage


\subsection{Equilibrio statico, elasticità e rotture}

\newpage


\subsection{Fluidi}

\newpage


\subsection{Oscillazioni}

\end{document}
