\documentclass[a4paper,12pt]{article}

% Pacchetti per la matematica e simboli
\usepackage{amsmath}    % Formule matematiche
\usepackage{amssymb}    % Simboli matematici
\usepackage{amsfonts}   % Font matematici
\usepackage{mathrsfs}   % Scrittura calligrafica matematica
\usepackage{amsthm}     % Teoremi e definizioni

% Pacchetti per grafici
\usepackage{graphicx}   % Inclusione grafica
\usepackage{tikz}       % Disegni e grafici
\usepackage{pgfplots}   % Grafici avanzati
\pgfplotsset{compat=1.18} % Imposta compatibilità per pgfplots

% Pacchetti per la gestione degli indici e riferimenti
\usepackage{hyperref}   % Riferimenti ipertestuali
\hypersetup{
    colorlinks=true,    % Colori per i link
    linkcolor=darkblue, % Colore dei link interni
    citecolor=darkblue, % Colore dei riferimenti alle citazioni
    filecolor=darkblue, % Colore dei link ai file
    urlcolor=darkblue,  % Colore degli URL
    pdftitle={Algebra elementare},
    pdfpagemode=UseOutlines
}
\usepackage{tocbibind}  % Include l'indice e la bibliografia nel sommario
\usepackage{fancyhdr}   % Intestazioni e piè di pagina personalizzati
\usepackage{bookmark}   % Gestione segnalibri PDF

% Configurazione per intestazioni e piè di pagina
\setlength{\headheight}{15pt} % Altezza dell'intestazione
\addtolength{\topmargin}{-10pt} % Riduzione del margine superiore

\pagestyle{fancy}
\fancyhf{}  % Pulisce intestazione e piè di pagina
\fancyhead[L]{Algebra elementare}  % Intestazione sinistra
\fancyhead[R]{\today}                      % Intestazione destra
\fancyfoot[C]{\thepage}                    % Numero di pagina al centro

% Impostazioni dei margini
\usepackage{geometry}
\geometry{
    left=1.2in,
    right=1.2in,
    top=1in,
    bottom=1.2in
}

% Gestione delle sezioni e sottosezioni
\usepackage{titlesec}
\titleformat{\section}[block]{\large\scshape}{\thesection}{1em}{} % Titoli sezioni maiuscoli
\titleformat{\subsection}[block]{\normalsize\bfseries}{\thesubsection}{1em}{}
\titleformat{\subsubsection}[block]{\normalsize\itshape}{\thesubsubsection}{1em}{}
\titleformat{\paragraph}[runin]{\normalsize\bfseries}{\theparagraph}{1em}{}[:]
\titleformat{\subparagraph}[runin]{\normalsize\itshape}{\thesubparagraph}{1em}{}[:]

% Colore e aspetto
\usepackage{xcolor}
\definecolor{darkblue}{rgb}{0.0, 0.2, 0.6}
\definecolor{gray}{rgb}{0.5, 0.5, 0.5}

% Teoremi e definizioni formali
\newtheoremstyle{mystyle} % Definisci lo stile del teorema
  {10pt} % Spaziatura sopra
  {10pt} % Spaziatura sotto
  {\itshape} % Corpo del teorema in corsivo
  {} % Indentazione del numero
  {\bfseries} % Font del titolo
  {}     % Punteggiatura dopo il titolo
  {\newline} % Spazio dopo il titolo
  {}     % Stile del testo

\theoremstyle{mystyle}
\newtheorem{theorem}{Teorema}[section]
\newtheorem{definition}[theorem]{Definizione}
\newtheorem{lemma}[theorem]{Lemma}
\newtheorem{corollary}[theorem]{Corollario}
\newtheorem{proposition}[theorem]{Proposizione}

% Pacchetti aggiuntivi
\usepackage{enumitem}   % Gestione elenchi personalizzati
\usepackage{multicol}   % Colonne multiple
\usepackage{booktabs}   % Tabelle formattate professionalmente
\usepackage{caption}    % Migliore gestione delle didascalie
\captionsetup{
    labelfont=bf,
    font=small,
    labelsep=colon
}

% Impostazioni di line spacing
\usepackage{setspace}
\onehalfspacing  % Interlinea 1.5 per migliorare la leggibilità

% Informazioni del documento
\title{\textbf{Algebra elementare}}
\author{\textit{Oudeys}}
\date{\today}

% Inizio del documento
\begin{document}

\maketitle

\begin{abstract}
    Questo documento contiene un formulario tecnico-scientifico di Alegebra elementare, pensato per studenti universitari e ricercatori. Esso raccoglie teoremi, definizioni, e strumenti matematici essenziali.
\end{abstract}

\tableofcontents
\newpage

\section{Insiemistica}


\section{Calcolo letterale}
\subsection{Frazioni}
\subsection{Proporzioni}
\subsection{Potenze}
\subsection{Radicali}
\subsection{Logaritmi}
\subsection{Esponenziali}

\subsection{Regole di scomposizione polinomi}
\begin{proposition}[Somma per differenza]
    \[(a-b) \cdot (a+b) = a^2 - b^2 \]
\end{proposition}

\begin{proposition}[Quadrato]
    \[(a \pm b)^2 = a^2 \pm 2ab + b^2  \]\\

    \[(a+b+c)^2 = a^2 + b^2 + c^2 + 2ab + 2bc + 2ac\]\\

    \[(a_1+a_2+ \ldots + a_n)^2 = a_1^2 + a_2^2 + \ldots + a_n^2 + 2a_1a_2 + \ldots + 2a_1a_n + 2a_2a_3 + \ldots + 2a_{n-1}a_n \]
\end{proposition}

\begin{proposition}[Cubo binomio]
    \[(a \pm b)^3 = a^3 \pm 3a^2b + 3ab^2 \pm b^3\]
\end{proposition}

\begin{proposition}[Potenza n-esima binomio]
    \[(a+b)^n = \binom{n}{0}a^n + \binom{n}{1}a^{n-1}b + \binom{n}{2}a^{n-2}b^2 + \ldots + \binom{n}{n-1}ab^{n-1} + \binom{n}{n}b^n \]
\end{proposition}

\begin{proposition}[Differenza di quadrati]
    \[a^2 - b^2 = (a-b) \cdot (a+b)\]
\end{proposition}

\subsubsection{Sviluppo del quadrato}

\[a^2 \pm 2ab + b^2 = (a \pm b)^2\]\\

\[a^2 + b^2 + c^2 + 2ab + 2ac + 2bc = (a + b + c)^2 \]

\subsubsection{Sviluppo del cubo di un binomio}

\[a^3 + 3a^2b + 3ab^2 + b^3 = (a + b)^3\]

\subsubsection{Somma di cubi}

\[a^3 + b^3 = (a+b) \cdot (a^2-ab+b^2)\]

\subsubsection{Differenza di cubi}
\[a^3 - b^3 = (a-b) \cdot (a^2 + ab + b^2)\]

\subsubsection{Differenza di potenze con uguale esponente}

\[a^n - b^n = (a-b) \cdot (a^{n-1} + a^{n-2}b + a^{n-3}b^2 + \ldots + ab^{n-2} + b^{n-1})  \]


\subsubsection{Differenza di potenze con uguale esponente pari}
\[a^n - b^n = (a+b) \cdot (a^{n-1} - a^{n-2}b + a^{n-3}b^2 - \ldots + ab^{n-2} - b^{n-1})\]

\subsubsection{Differenza di potenze con uguale esponente dispari}
\[a^n - b^n = (a + b) \cdot (a^{n-1} - a^{n-2}b + a^{n-3}b^2 - \ldots - ab^{n-2} + b^{n-1})\]

\subsubsection{Trinomio caratteristico di secondo grado}
\[s= a + b, \quad p = a \cdot b \]\\

\[x^2 + sx + p = (x+a) \cdot (x+b) \]

\subsubsection{Trinomio di secondo grado}
\[ax^2 + bx + c = a \cdot (x - x_1) \cdot (x - x_2)\]


\newpage

\subsection{Divisone fra polinomi}

\subsubsection{Regole di divisibilità somma/differenza potenze aventi stesso esponente}
\[a^n - b^n = (a-b) \cdot (a^{n-1} + a^{n-2}b + a^{n-3}b^2 + \ldots + ab^{n-2} + b^{n-1}) \]\\

\[a^n-b^n= (a+b) \cdot (a^{n-1}-a^{n-2}b + a^{n-3}b^2 - \ldots + ab^{n-2} - b^{n-1})\]\\

\[a^n + b^n = (a+b) \cdot (a^{n-1} - a^{n-2}b + a^{n-3}b^2 - \ldots -ab^{n-2} + b^{n-1})\]


\newpage


\section{Calcolo combinatorio}

\subsection{Fattoriale e semifattoriale}
\[n! = n \cdot (n-1) \cdot (n-2) \cdot \ldots \cdots 2 \cdots 1 \]\\
\(0! = 1\)\\
\((n+1)! = (n+1) \cdots n! \)\\
\(n!!=
\begin{cases}
    1 \cdot 3 \cdot 5 \cdot \ldots \cdots n \Leftrightarrow n \, \text{ dispari} \\
    2 \cdot 4 \cdot 6 \cdot \ldots \cdot n \Leftrightarrow n \text{ pari}
\end{cases} \)

\subsection{Coefficiente binomiale}
\[\binom{n}{k} = \frac{n!}{(n-k)! \cdot k!} = \frac{n(n-1)(n-2) \cdots (n-k+1)}{k!} \]
\((a+b)^n = \binom{n}{0}a^n + \binom{n}{1}a^{n-1}b + \binom{n}{2}a^{n-2}b^2 + \ldots + \binom{n}{n-1}ab^{n-1}+\binom{n}{n}b^n \) \\
\(\binom{n}{0}=\binom{n}{n}=1 \) \\
\(\binom{n}{k}=\binom{n}{n-k} \) \\
\(\binom{n}{k}+\binom{n}{k+1}=\binom{n+1}{k+1} \) \\
\(\binom{n}{k+1}=\binom{n}{k} \cdot \frac{n-k}{k+1}\)

\newpage

\section{Insiemi numerici}

\subsection{Numeri naturali}
\subsection{Numeri interi}
\subsection{Numeri razionali}
\subsection{Numeri irrazionali}
\subsection{Numeri reali}

\subsection{Numeri complessi}


\subsubsection{Potenze unità immaginaria}
\(i^2=-1 \)\\
\(i^3=i^2 \cdot i = -i \)\\
\(i^4 = i^2 \cdot i^2 = 1 \)\\
\(i^5 = i^4 \cdot i = i \)\\
\(i^6 = -1 \)\
\(i^7 = -i \)\\
\(i^8 = 1 \)\\
\(i^9=i \)

\subsubsection{Numeri immaginari}
\((ai)^2 = a^2i^2 = a^2 \cdot (-1) = - a^2\)

\subsubsection{Rappresentazione algebrica numeri complessi}
\[z = a + ib\]

\subsubsection{Addizione}
\(z_1+z_2=(x_1+iy_1)+(x_2+iy_2) =(x_1+x_2)+i(y_1+y_2)\)

\subsubsection{Moltiplicazione}
\(z_1 \cdot z_2 = (x_1+iy_1)(x_2+iy_2) \)\\
\((x_1x_2-y_1y_2)+i(x_1y_2 + y_1x_2) \)

\subsubsection{Parte reale}
\(\Re (z) = \frac{z + \overline{z}}{2} =x \)

\subsubsection{Parte immaginaria}
\(\Im  (z) = \frac{z - \overline{z}}{2} = y \)


\subsubsection{Reciproco complesso}
\(\frac{1}{z} = \frac{1}{x+iy} = \frac{x-iy}{(x+iy)(x-iy)} = \frac{x-iy}{x^2+y^2} \)

\subsubsection{Complesso coniugato}
\(\overline z = x-iy \)\\
\(z \overline z = x^2 + y^2 \)\\
\(\overline{z_1+z_2} = \overline {z_1} + \overline {z_2} \)\\
\(\overline {z_1 z_2} = \overline {z_1} \overline {z_2} \)

\subsubsection{Modulo}
\(\rho =\lvert z \rvert = \lvert x + iy \rvert = \sqrt{x^2 + y^2}\)

\subsubsection{Argomento}
\(\Re z = \rho \cos (\theta) \)\\
\(\Im z = \rho \sin (\theta) \)\\
\(\sin (\theta) =\frac{y}{\rho} \)\\
\(\cos (\theta) = \frac{x}{\rho} \)\\
\(\tan (\theta) = \frac{y}{x} \forall a \neq 0 \)

\subsubsection{Rappresentazione trigonometrica}
\[z= \rho(\cos(\theta + i \sin \theta) \]

\(z_1 \cdot z_2 = \rho_1 (\cos \theta_1 + i \sin \theta _1) \cdot \rho_2 (\cos \theta _2 + i \sin \theta _2 ) = \rho_1 \rho_2 [(\cos \theta_1 \cos \theta_2 - \sin \theta_1 \sin \theta_2)+i(\sin \theta_1 \cos \theta_2 + \sin \theta_2 \cos \theta_1)] =\rho_1 \rho_2 [\cos(\theta_1 + \theta_2) + i \sin(\theta_1 + \theta_2)] \)

\subsubsection{Quadrato di un numero complesso}
\(z^2 = \rho^2 (\cos 2 \theta + i \sin 2 \theta)\)

\subsubsection{Formula di De Moivre per la potenza n-esima}
\[z^n = \rho^n (\cos n \theta + i\sin n\theta)\]

\subsubsection{Rappresentazione esponenziale}
\[z = \rho e^{i \theta} = \rho (\cos \theta + i \sin \theta)\]

\(e^{2k \pi i} = 1 \)\\
\(e^{i \theta + 2k \pi i} = e^i \)\\
\(\lvert e^{i \theta} = 1 \rvert \)\\
\(e^{\pi i} = -1 \)\\
\(e^{\pi i / 2} = i \)\\
\(e^{3 \pi i /2} = - i = (-1 + i) /\sqrt 2 \)\\
\(e^{-i \theta}= \cos(-\theta + i \sin (-\theta) = \cos \theta - i \sin \theta \)\\
\(\cos \theta = \frac{e^{i \theta} + e^{-i \theta}}{2} \)\\
\(\sin \theta = \frac{e^{i \theta}-e^{-i \theta}}{2i} \)

\subsubsection{Radici complesse}
\(z^n = w \Rightarrow z_k = \sqrt[n]{\lvert w \rvert} e^{i(\theta w + 2k \pi)/n} \)\\
\[w_k = \rho^{1/n} \left[ \cos \frac{\theta + 2k \pi}{n} + i \sin \frac{\theta + 2k \pi}{n} \right] \]

\subsubsection{Radici equazione quadratica}
\(a_x^2 + bz + c = 0 \Leftrightarrow z = \frac{-b \pm \sqrt{b^2 - 4ac}}{2a} \)



\newpage

\section{Equazioni algebriche}
\section{Disequazioni algebriche}

\section{Funzioni}

\subsection{Funzione simmetrica}
\(\text{Pari}\Leftrightarrow f(-x)=f(x)\)\\
\(\text{Dispari}\Leftrightarrow f(-x)=-f(x)\)\\
\(\text{Periodica}\Leftrightarrow f(x+T)=f(x)\)

\subsection{Funzione elementari}
\subsubsection{Funzioni lineari}

\subsubsection{Funzioni quadratiche}

\subsubsection{Funzioni potenza}
\(\forall n \in \mathbb N , n \text{ pari}\)\\
\begin{equation}
    f(x)=x^n
\end{equation}
\(dom f = \mathbb R\)\\
\(im f = [0, + \infty)\)
\(f'<0 \in (- \infty, 0]\)\\
\(f'>0 \in [0, + \infty)\)\\
\(f \text{pari}\)\\

\(\forall n \in \mathbb N, n \text{ dispari}\)\\
\begin{equation}
    f(x)=x^n
\end{equation}
\(dom f = \mathbb R\)\\
\(im f = \mathbb R\)\\
\(f'>0 \in \mathbb R\)\\
\(f \text{ dispari}\)\\

\(\forall n \in \mathbb N, n \text{dispari}\)\\
\begin{equation}
    f(x)=x^{-n}
\end{equation}
\(dom f= \mathbb R\backslash \{0\}\)\\
\(im f = \mathbb R \backslash \{0\}\)\\
\(f'<0 \in (-\infty,0)\)\\
\(f'>0 \in (0, + \infty)\)\\
\(f \text{dispari}\)\\

\(\forall n \in \mathbb N, n \geq 2 \text{pari}\)\\
\begin{equation}
    f(x) = x^{-n}
\end{equation}
\(dom f = \mathbb R \backslash \{0\}\)\\
\(im f = (0, + \infty)\)\\
\(f'<0 \in (-\infty,0)\)\\
\(f'>0 \in (0,+\infty)\)\\
\(f \text{pari}\)\\

\(\forall n \in \mathbb N, n \geq 2 \text{pari}\)\\
\begin{equation}
    f(x)=x^{1/n}
\end{equation}
\(dom f= [0, + \infty)\)\\
\(im f = [0,+\infty)\)\\
\(f'>0 \in [0. + \infty)\)\\

\(n \in \mathbb N, n \text{dispari}\)\\
\begin{equation}
    f(x)= x^{1/n}
\end{equation}
\(dom f= \mathbb R\)\\
\(im f = \mathbb R\)\\
\(f'> \in \mathbb R\)\\
\(f \text{dispari}\)

\(\forall \alpha \in \mathbb R, \alpha >0\)
\begin{equation}
    f(x)=x^\alpha
\end{equation}
\(dom f= [0, + \infty)\)\\
\(im f = [0, + \infty)\)\\
\(f'> 0 \in [0, + \infty)\)\\

\(\forall \alpha \in \mathbb R, \alpha < 0\)\\
\begin{equation}
    f(x)=x^\alpha
\end{equation}
\(dom f=(0, + \infty)\)\\
\(im f = (0, + \infty)\)\\
\(f' < 0 \in (0, + \infty)\)


\subsubsection{Funzioni esponenziali}
\(\forall a\in \mathbb R,a >0, a \neq 1\)\\
\begin{equation}
    f(x)=a^x
\end{equation}
\(dom f = \mathbb R\)\\
\(im f = (0, + \infty)\)\\
\(f'>0 \in \mathbb R \Leftrightarrow a >1\)\\
\(f'<0 \in \mathbb R \Leftrightarrow a \in (0,1)\)

\subsubsection{Funzioni logaritmiche}
\(\forall a \in \mathbb R, a >0, a \neq 1\)
\begin{equation}
    f(x)= \log_a x
\end{equation}
\(dom f= (0,+\infty)\)\\
\(im f = \mathbb R\)\\
\(f'>0 \in \mathbb R^+ \Leftrightarrow a >1\)\\
\(f'<0 \in \mathbb R^+ \Leftrightarrow a \in (0,1)\)

\subsubsection{Funzioni trigonometriche}
\begin{equation}
    f(x)= \sin x
\end{equation}
\(dom f = \mathbb R\)
\(im f = [-1,1]\)\\
\(f'>0 \forall x \in[-\pi/2+2k\pi,\pi/2+2k\pi]\)\\
\(f'<0 \forall x \in [\pi/2+2k\pi, 3\pi/2 + 2k\pi]\)\\

\begin{equation}
    f(x)= \cos x
\end{equation}
\(dom f = \mathbb R\)
\(im f = [-1,1]\)\\
\(f'>0 \forall x \in [\pi + 2k \pi, 2(k+1)\pi]\)\\
\(f'<0 \forall x \in [2k\pi, \pi + 2k\pi]\)

\begin{equation}
    f(x) = \tan x
\end{equation}
\(dom f = \mathbb R \backslash \{\pi / 2 + k \pi, k \in \mathbb Z\}\)\\
\(im f = \mathbb R\)\\
\(f'>0 \forall x \in (-\pi/2+k\pi,\pi/2+k\pi)\)


\subsection{Potenze}
\(a^1=a\)\\
\(0^n=0 \forall x \neq 0\)\\
\(1^n = 1\)\\
\(a^0=1 \forall a \neq 0\)\\
\(a^m \cdot a^n = a^{m+n}\)\\
\(\frac{a^m}{a^n}=a^{m-n}\)\\
\((a^m)^n=a^{m \cdot n}\)\\
\((a\cdot b \cdot c)^n=a^n \cdot b^n \cdot c^n\)\\
\(\left(\frac{a}{b} \right) ^n =\frac{a^n}{b^n}\)\\
\(a^{-n}=\frac{1}{a^n}\)\\
\(\left(\frac{a}{b}\right) ^{-n} = \left (\frac{b}{a}\right)^n\)

\subsection{Logaritmi}
\(\log_a (b \cdot c) = \log a_a b + \log_a c\)\\
\(\log_a \left (\frac{b}{c}\right ) = \log_a b - \log_a c\)\\
\(\log_a (b)^n = n \cdot \log_a b\)\\
\(\log_a \sqrt[n]{b^m} = \frac{m}{n} \cdot \log_a b\)\\
\(\log_a b = \frac{1}{\log_b a}\)\\
\(\log_{\frac{1}{a}} b = -\log_a b \)\\
\(\log_a b = \frac{\log_c b}{\log_c a}\)

\subsection{Iperboliche}

\(\sinh x = \frac{e^x-e^{-x}}{2}\) \\
\(\cosh x = \frac{e^x + e^{-x}}{2}\) \\
\(\tanh x = \frac{e^x-e^{-x}}{e^x+e^{-x}}\) \\
\(\coth x = \frac{e^x+e^{-x}}{e^x-e^{-x}}\)

\subsection{Successioni}

\section{Curve}

\section{Superfici}

\end{document}

