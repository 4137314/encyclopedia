\documentclass[a4paper,12pt]{article}

% Pacchetti per la matematica e simboli
\usepackage{amsmath}    % Formule matematiche
\usepackage{amssymb}    % Simboli matematici
\usepackage{amsfonts}   % Font matematici
\usepackage{mathrsfs}   % Scrittura calligrafica matematica
\usepackage{amsthm}     % Teoremi e definizioni

% Pacchetti per grafici
\usepackage{graphicx}   % Inclusione grafica
\usepackage{tikz}       % Disegni e grafici
\usepackage{pgfplots}   % Grafici avanzati
\pgfplotsset{compat=1.18} % Imposta compatibilità per pgfplots

% Pacchetti per la gestione degli indici e riferimenti
\usepackage{hyperref}   % Riferimenti ipertestuali
\hypersetup{
    colorlinks=true,    % Colori per i link
    linkcolor=darkblue, % Colore dei link interni
    citecolor=darkblue, % Colore dei riferimenti alle citazioni
    filecolor=darkblue, % Colore dei link ai file
    urlcolor=darkblue,  % Colore degli URL
    pdftitle={Formulario di Analisi Reale},
    pdfpagemode=UseOutlines
}
\usepackage{tocbibind}  % Include l'indice e la bibliografia nel sommario
\usepackage{fancyhdr}   % Intestazioni e piè di pagina personalizzati
\usepackage{bookmark}   % Gestione segnalibri PDF

% Configurazione per intestazioni e piè di pagina
\setlength{\headheight}{15pt} % Altezza dell'intestazione
\addtolength{\topmargin}{-10pt} % Riduzione del margine superiore

\pagestyle{fancy}
\fancyhf{}  % Pulisce intestazione e piè di pagina
\fancyhead[L]{Formulario di Analisi Reale}  % Intestazione sinistra
\fancyhead[R]{\today}                      % Intestazione destra
\fancyfoot[C]{\thepage}                    % Numero di pagina al centro

% Impostazioni dei margini
\usepackage{geometry}
\geometry{
    left=1.2in,
    right=1.2in,
    top=1in,
    bottom=1.2in
}

% Gestione delle sezioni e sottosezioni
\usepackage{titlesec}
\titleformat{\section}[block]{\large\scshape}{\thesection}{1em}{} % Titoli sezioni maiuscoli
\titleformat{\subsection}[block]{\normalsize\bfseries}{\thesubsection}{1em}{}
\titleformat{\subsubsection}[block]{\normalsize\itshape}{\thesubsubsection}{1em}{}
\titleformat{\paragraph}[runin]{\normalsize\bfseries}{\theparagraph}{1em}{}[:]
\titleformat{\subparagraph}[runin]{\normalsize\itshape}{\thesubparagraph}{1em}{}[:]

% Colore e aspetto
\usepackage{xcolor}
\definecolor{darkblue}{rgb}{0.0, 0.2, 0.6}
\definecolor{gray}{rgb}{0.5, 0.5, 0.5}

% Teoremi e definizioni formali
\newtheoremstyle{mystyle} % Definisci lo stile del teorema
  {10pt} % Spaziatura sopra
  {10pt} % Spaziatura sotto
  {\itshape} % Corpo del teorema in corsivo
  {} % Indentazione del numero
  {\bfseries} % Font del titolo
  {}     % Punteggiatura dopo il titolo
  {\newline} % Spazio dopo il titolo
  {}     % Stile del testo

\theoremstyle{mystyle}
\newtheorem{theorem}{Teorema}[section]
\newtheorem{definition}[theorem]{Definizione}
\newtheorem{lemma}[theorem]{Lemma}
\newtheorem{corollary}[theorem]{Corollario}
\newtheorem{proposition}[theorem]{Proposizione}

% Pacchetti aggiuntivi
\usepackage{enumitem}   % Gestione elenchi personalizzati
\usepackage{multicol}   % Colonne multiple
\usepackage{booktabs}   % Tabelle formattate professionalmente
\usepackage{caption}    % Migliore gestione delle didascalie
\captionsetup{
    labelfont=bf,
    font=small,
    labelsep=colon
}

% Impostazioni di line spacing
\usepackage{setspace}
\onehalfspacing  % Interlinea 1.5 per migliorare la leggibilità

% Informazioni del documento
\title{\textbf{Probabilità e Statistica}}
\author{\textit{Oudeys}}
\date{\today}

% Inizio del documento
\begin{document}

\maketitle



\tableofcontents
\newpage

\section{Probabilità}

\subsection{Calcolo combinatorio}
\subsubsection{\texorpdfstring{Disposizioni di \(n\) elementi su \(k\) posti}{Disposizioni di n elementi su k posti}}
\(\forall \, n \geq k\)
\[
    \begin{aligned}
        D_{n,k} 
        &= n(n-1)(n-2) \ldots (n-k+1) \\
        &= \frac{n(n-1)(n-2) \ldots (n-k+1)(n-k)!}{(n-k)!} \\
        &= \frac{n!}{(n-k)!}
    \end{aligned}
\]

\subsubsection{Disposizioni con ripetizione}

\[
    D_{n,k}^r = n^k    
\]

\subsubsection{\texorpdfstring{Permutazioni di \(n\) elementi}{Permutazioni di n elementi}}
\[
    P_n = D_{n,n} = n!
\]

\subsubsection{\texorpdfstring{Combinazioni di \(n\) elementi presi \(k\) alla volta}{Combinazioni di n elementi presi k alla volta}}
\(\forall \, n>k\)

\[
    \begin{aligned}
        C_{n,k}
        & = \frac{D_{n,k}}{k!} \\
        & = \frac{n!}{k!(n-k)!}
    \end{aligned}
\]

\subsubsection{Combinazioni con ripetizione}
\(\forall \, n>1\)
\[
    C_{n,k}^r = \frac{(n+k-1)!}{k!(n-1)!}
\]

\subsubsection{\texorpdfstring{Ripartizioni di \(n\) elementi in \(m\) classi}{Riprtizioni di n elementi in m classi}}
\(\sum_{i=1}^{m} k_i = n\)

\[
    R_{n;k_1,\ldots,k_m} = \frac{n!}{k1! \cdots k_m!}
\]

\subsection{Impostazione assiomatica del calcolo delle probabilità}

\subsection{Legami stocastici tra eventi}
\subsubsection{Probabilità condizionata}
\(\forall \, \Pr\{A\}>0\)
\[
    \Pr\{A|B\} = \frac{\Pr\{A \cap B\}}{\Pr\{A\}}
\]

\subsubsection{Regola del prodotto}
\[
    \begin{aligned}
        \Pr\{A\cap B\}
        &=\Pr\{A\}\Pr\{B|A\} \\
        &=\Pr\{B\}\Pr\{A|B\}
    \end{aligned}
\]

\subsubsection{Indipendenza stocastica}
\[
    \begin{aligned}
        &\Pr\{A|B\} = \Pr\{A\} \\
        &\Leftrightarrow \\
        &\Pr\{B|A\} = \Pr\{B\} \\
        &\Leftrightarrow \\
        &\Pr\{A \cap B\} = \Pr\{A\}\Pr\{B\}
    \end{aligned}
\]

\subsection{Formule fondamentali}
\subsubsection{Probabilità dell'unione di eventi non incompatibili}
\[
    \Pr\{A \cup B\} = \Pr\{A\} + \Pr\{B\} - \Pr\{A \cap B\}
\]

\subsubsection{Regola della fattorizzazione}
\[
    \begin{aligned}
    \Pr\{A\}
    &= \Pr\{A \cap C\} + \Pr\{A \cap \overline C\} \\
    &= \Pr\{A|C\} \Pr\{C\} + \Pr\{A|\overline C\} \Pr\{\overline C\}
    \end{aligned}
\]

\subsubsection{Teorema di Bayes - Regola della probabilità delle cause}
\(\forall \, \Pr \{E_i\}>0, \, \Pr\{A\}>0\)

\[
    \Pr \{E_j | A\} = \frac{\Pr \{E_j\} \Pr \{A|E_j\}}{\sum_{i=1}^{n} \Pr \{E_i\} \Pr\{A|E_i\}}
\]

\section{Variabili aleatorie}

\subsection{Variabili aleatorie monodimensionali}
\subsubsection{\texorpdfstring{Funzione distribuzione - \(Cdf\)}{Funzione distribuzione - Cdf}}
\[
    \begin{aligned}
        F(x)
        & = \Pr \{X \leq x\} \\
        & = F(x^+) \\
        & = \lim_{\epsilon \rightarrow 0^+} F(x+\epsilon)
    \end{aligned}
\] \\
\(\forall \, x_2 > x_1 : F(x_2) \geq F(x_1)\) \\
\(\lim_{x \rightarrow -\infty} F(x) = 0\) \\
\(\lim_{x \rightarrow + \infty} F(x) = 1\) \\
\(0 \leq F(x) \leq 1\)

\subsubsection{\texorpdfstring{Funzione massa di probabilità - \(pmf\)}{Funzione massa di probabilità - pmf}}

\[
    \begin{aligned}
        \Pr\{X=x\}
        & = F(x) - \lim_{\epsilon \rightarrow 0^+} F(x-\epsilon) \\
        & = F(x) - F(x^-)
    \end{aligned}
\]

\subsubsection{\texorpdfstring{Funzione densità di probabilità - \(pdf\)}{Funzione densità di probabilità - pdf}}

\[
    \begin{aligned}
        f(x)
        &=\lim_{\Delta x \rightarrow 0} \frac{F(x+\Delta x) - F(x)}{\Delta x} \\
        & = \frac{d}{dx} F(x)
    \end{aligned}
\] \\

\(F(x_1) = \int_{X \leq x_1} f(x) dx\) \\
\(
    \begin{aligned}
        \forall x_1 < x_2 : \Pr\{x_1 <X \leq x_2\}
        & =  F(x_2) - F(x_1) \\
        & = \int_{x_1}^{x_2} f(x) dx
    \end{aligned}
\) \\
\(\int_{-\infty}^{+\infty} f(x) dx = 1 \) \\
\(\forall x \in \mathbb R : f(x) \geq 0\)

\subsection{Variabili aleatorie bidimensionali}

\section{Modelli di variabili aleatorie discrete}

\section{Modelli di variabili aleatorie continue}


\end{document}