\section{Fondamenti di Chimica Inorganica}
\subsection{Teoria del legame chimico}
\subsection{Geometria molecolare}
\subsection{Teoria degli orbitali molecolari}

\section{Acidi, Basi e Sali}
\subsection{Teoria degli acidi e delle basi}
\subsection{Equilibri acido-base}
\subsection{Proprietà dei sali}

\section{Chimica dei Metalli di Transizione}
\subsection{Proprietà generali}
\subsection{Complessi di coordinazione}
\subsection{Reattività dei metalli di transizione}

\section{Chimica dei Non-Metalli}
\subsection{Proprietà generali}
\subsection{Ossidi e anidridi}
\subsection{Reattività dei non-metalli}

\section{Chimica dei Metalli Alcalini e Alcalino-terrosi}
\subsection{Proprietà generali}
\subsection{Reattività degli alcalini}
\subsection{Reattività degli alcalino-terrosi}

\section{Chimica dei Lantanidi e Attinidi}
\subsection{Proprietà generali}
\subsection{Reattività dei lantanidi}
\subsection{Reattività degli attinidi}

\section{Chimica degli Elementi di Blocco p}
\subsection{Proprietà generali}
\subsection{Reattività degli elementi del blocco p}
\subsection{Composti e applicazioni}

\section{Chimica dello Stato Solido}
\subsection{Struttura cristallina}
\subsection{Proprietà dei solidi}
\subsection{Difetti nei solidi}

\section{Tecniche di Analisi Strutturale}
\subsection{Diffrazione a raggi X}
\subsection{Spettroscopia}
\subsection{Microscopia elettronica}

\section{Applicazioni della Chimica Inorganica}
\subsection{Materiali inorganici}
\subsection{Catalisi}
\subsection{Chimica ambientale}
