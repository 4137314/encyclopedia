\section{Fondamenti di Chimica Organica}
\subsection{Struttura e legame del carbonio}
\subsection{Formule chimiche e strutture}
\subsection{Nomenclatura degli idrocarburi e dei loro derivati}
\subsection{Isomeria e stereochimica}

\section{Idrocarburi}
\subsection{Alifatici: alcani, alcheni, alchini}
\subsection{Ciclici: cicloalcani, cicloalcheni}
\subsection{Aromatici: benzene e composti aromatici}

\section{Gruppi Funzionali e Reattività}
\subsection{Alogenuri alchilici}
\subsection{Alcoli, fenoli ed eteri}
\subsection{Aldeidi e chetoni}
\subsection{Acidi carbossilici e derivati}
\subsection{Ammine e composti contenenti azoto}
\subsection{Tioeteri, solfuri e solfossidi}

\section{Meccanismi di Reazione}
\subsection{Reazioni di sostituzione nucleofila}
\subsection{Reazioni di eliminazione}
\subsection{Reazioni di addizione elettrofila e nucleofila}
\subsection{Reazioni radicaliche}
\subsection{Reazioni pericicliche}

\section{Sintesi Organica}
\subsection{Metodi di sintesi di composti organici}
\subsection{Sintesi a tappe}
\subsection{Protezione e deprotezione di gruppi funzionali}
\subsection{Catalisi in chimica organica}

\section{Chimica Organometallica}
\subsection{Composti organometallici e loro reattività}
\subsection{Catalizzatori organometallici}
\subsection{Applicazioni industriali}

\section{Chimica dei Polimeri}
\subsection{Tipi di polimeri e loro sintesi}
\subsection{Struttura e proprietà dei polimeri}
\subsection{Applicazioni dei polimeri}

\section{Biochimica Organica}
\subsection{Struttura e funzione delle biomolecole}
\subsection{Carboidrati, lipidi, proteine e acidi nucleici}
\subsection{Enzimi e catalisi biologica}
\subsection{Metabolismo e bioenergetica}