\documentclass[a4paper,12pt]{article}

% Pacchetti per la matematica e simboli
\usepackage{amsmath}    % Formule matematiche
\usepackage{amssymb}    % Simboli matematici
\usepackage{amsfonts}   % Font matematici
\usepackage{mathrsfs}   % Scrittura calligrafica matematica
\usepackage{amsthm}     % Teoremi e definizioni

% Pacchetti per grafici
\usepackage{graphicx}   % Inclusione grafica
\usepackage{tikz}       % Disegni e grafici
\usepackage{pgfplots}   % Grafici avanzati
\pgfplotsset{compat=1.18} % Imposta compatibilità per pgfplots

% Pacchetti per la gestione degli indici e riferimenti
\usepackage{hyperref}   % Riferimenti ipertestuali
\hypersetup{
    colorlinks=true,    % Colori per i link
    linkcolor=darkblue, % Colore dei link interni
    citecolor=darkblue, % Colore dei riferimenti alle citazioni
    filecolor=darkblue, % Colore dei link ai file
    urlcolor=darkblue,  % Colore degli URL
    pdftitle={Formulario di Analisi Reale},
    pdfpagemode=UseOutlines
}
\usepackage{tocbibind}  % Include l'indice e la bibliografia nel sommario
\usepackage{fancyhdr}   % Intestazioni e piè di pagina personalizzati
\usepackage{bookmark}   % Gestione segnalibri PDF

% Configurazione per intestazioni e piè di pagina
\setlength{\headheight}{15pt} % Altezza dell'intestazione
\addtolength{\topmargin}{-10pt} % Riduzione del margine superiore

\pagestyle{fancy}
\fancyhf{}  % Pulisce intestazione e piè di pagina
\fancyhead[L]{Formulario di Elaborazione dei Segnali}  % Intestazione sinistra
\fancyhead[R]{\today}                      % Intestazione destra
\fancyfoot[C]{\thepage}                    % Numero di pagina al centro

% Impostazioni dei margini
\usepackage{geometry}
\geometry{
    left=1.2in,
    right=1.2in,
    top=1in,
    bottom=1.2in
}

% Gestione delle sezioni e sottosezioni
\usepackage{titlesec}
\titleformat{\section}[block]{\large\scshape}{\thesection}{1em}{} % Titoli sezioni maiuscoli
\titleformat{\subsection}[block]{\normalsize\bfseries}{\thesubsection}{1em}{}
\titleformat{\subsubsection}[block]{\normalsize\itshape}{\thesubsubsection}{1em}{}
\titleformat{\paragraph}[runin]{\normalsize\bfseries}{\theparagraph}{1em}{}[:]
\titleformat{\subparagraph}[runin]{\normalsize\itshape}{\thesubparagraph}{1em}{}[:]

% Colore e aspetto
\usepackage{xcolor}
\definecolor{darkblue}{rgb}{0.0, 0.2, 0.6}
\definecolor{gray}{rgb}{0.5, 0.5, 0.5}

% Teoremi e definizioni formali
\newtheoremstyle{mystyle} % Definisci lo stile del teorema
  {10pt} % Spaziatura sopra
  {10pt} % Spaziatura sotto
  {\itshape} % Corpo del teorema in corsivo
  {} % Indentazione del numero
  {\bfseries} % Font del titolo
  {}     % Punteggiatura dopo il titolo
  {\newline} % Spazio dopo il titolo
  {}     % Stile del testo

\theoremstyle{mystyle}
\newtheorem{theorem}{Teorema}[section]
\newtheorem{definition}[theorem]{Definizione}
\newtheorem{lemma}[theorem]{Lemma}
\newtheorem{corollary}[theorem]{Corollario}
\newtheorem{proposition}[theorem]{Proposizione}

% Pacchetti aggiuntivi
\usepackage{enumitem}   % Gestione elenchi personalizzati
\usepackage{multicol}   % Colonne multiple
\usepackage{booktabs}   % Tabelle formattate professionalmente
\usepackage{caption}    % Migliore gestione delle didascalie
\captionsetup{
    labelfont=bf,
    font=small,
    labelsep=colon
}

% Impostazioni di line spacing
\usepackage{setspace}
\onehalfspacing  % Interlinea 1.5 per migliorare la leggibilità

% Informazioni del documento
\title{\textbf{Logica elementare}}
\author{\textit{Oudeys}}
\date{\today}

% Inizio del documento
\begin{document}

\maketitle


\tableofcontents
\newpage

\section{Logica proposizionale}
\subsection{Operazioni logiche}

\subsubsection{Negazione}
\begin{table}[h]
    \centering
    \begin{tabular}{|c|c|}
        \hline
        \textbf{\(A\)} & \textbf{\(\neg A\)} \\
        \hline
        0 & 1 \\
        \hline
        1 & 0 \\
        \hline
    \end{tabular}
\end{table}

\subsubsection{Congiunzione}
\begin{table}[h]
    \centering
    \begin{tabular}{|c|c|c|}
        \hline
        \textbf{\(A\)} & \textbf{\(B\)} & \textbf{\(A \land B\)} \\
        \hline
        0 & 0 & 0 \\
        \hline
        0 & 1 & 0 \\
        \hline
        1 & 0 & 0 \\
        \hline
        1 & 1 & 1 \\
        \hline
    \end{tabular}
\end{table}

\subsubsection{Disgiunzione inclusiva}
\begin{table}[h]
    \centering
    \begin{tabular}{|c|c|c|}
        \hline
        \textbf{\(A\)} & \textbf{\(B\)} & \textbf{\(A \lor B\)} \\
        \hline
        0 & 0 & 0 \\
        \hline
        0 & 1 & 1 \\
        \hline
        1 & 0 & 1 \\
        \hline
        1 & 1 & 1 \\
        \hline
    \end{tabular}
\end{table}

\newpage

\subsubsection{Disgiunzione esclusiva}
\begin{table}[h]
    \centering
    \begin{tabular}{|c|c|c|}
        \hline
        \textbf{\(A\)} & \textbf{\(B\)} & \textbf{\(A \oplus B\)} \\
        \hline
        0 & 0 & 0 \\
        \hline
        0 & 1 & 1 \\
        \hline
        1 & 0 & 1 \\
        \hline
        1 & 1 & 0 \\
        \hline
    \end{tabular}
\end{table}

\subsubsection{Condizionale}
\begin{table}[h]
    \centering
    \begin{tabular}{|c|c|c|}
        \hline
        \textbf{\(A\)} & \textbf{\(B\)} & \textbf{\(A \Rightarrow B\)} \\
        \hline
        0 & 0 & 1 \\
        \hline
        0 & 1 & 1 \\
        \hline
        1 & 0 & 0 \\
        \hline
        1 & 1 & 1 \\
        \hline
    \end{tabular}
\end{table}

\subsubsection{Bicondizionale}
\begin{table}[h]
    \centering
    \begin{tabular}{|c|c|c|}
        \hline
        \textbf{\(A\)} & \textbf{\(B\)} & \textbf{\(A \Leftrightarrow B\)} \\
        \hline
        0 & 0 & 1 \\
        \hline
        0 & 1 & 0 \\
        \hline
        1 & 0 & 0 \\
        \hline
        1 & 1 & 1 \\
        \hline
    \end{tabular}
\end{table}

\section{Inferenze}
\subsection{Modus ponens}
\subsection{Modus tollens}
\subsection{Sillogismo}
\subsection{Contraddizione}

\section{Logica dei predicati}
\subsection{Quantificatori}
\subsubsection{Quantificatore universale}
\subsubsection{Quantificatore essenziale}

\section{Leggi logiche}

\begin{proposition}[Identità]
    \[
        A \Rightarrow A
    \]
\end{proposition}

\begin{proposition}[Doppia negazione]
    \[
        A \Leftrightarrow \neg \neg A
    \]
\end{proposition}

\begin{proposition}[Commutativa di \(\land\)]
    \[
        A \land B \Leftrightarrow B \land A
    \]
\end{proposition}

\begin{proposition}[Associativa di \(\land\)]
    \[
        (A \land B) \land C \Leftrightarrow A \land (B \land C)
    \]
\end{proposition}

\begin{proposition}[Commutativa di \(\lor\)]
    \[
        A \lor B \Leftrightarrow B \lor A
    \]
\end{proposition}

\begin{proposition}[Associativa di \(\lor\)]
    \[
        (A \lor B) \lor C \Leftrightarrow A \lor (B \lor C)
    \]
\end{proposition}

\begin{proposition}[Indepotenza di \(\land\)]
    \[
        A \land A = A
    \]
\end{proposition}

\begin{proposition}[Indepotenza di \(\lor\)]
    \[
        A \lor A = A
    \]
\end{proposition}

\begin{proposition}[Distributiva di \(\land\)]
    \[
        A \land (B \lor C) = (A \land B) \lor (A \land C)
    \]
\end{proposition}

\begin{proposition}[Distributiva di \(\lor\)]
    \[
        A \lor (B \land C) = (A \lor B) \land (A \lor C)
    \]
\end{proposition}

\begin{proposition}[Assorbimento]
    \[
        A \land (A \lor B) \Leftrightarrow A
    \]
\end{proposition}

\begin{proposition}[Assorbimento]
    \[
        A \lor (A \land B) \Leftrightarrow A
    \]
\end{proposition}

\begin{proposition}[De Morgan]
    \[
        \neg (A \land B) \Leftrightarrow (\neg A \lor \neg B)
    \]
\end{proposition}

\begin{proposition}[De Morgan]
    \[
        \neg(A \lor B) \Leftrightarrow (\neg(A \land \neg B))
    \]
\end{proposition}

\begin{proposition}[Terzo escluso]
    \[
        A \lor \neg A
    \]
\end{proposition}

\end{document}