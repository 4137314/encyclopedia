\documentclass[a4paper,12pt]{article}

% Pacchetti per la matematica e simboli
\usepackage{amsmath}    % Formule matematiche
\usepackage{amssymb}    % Simboli matematici
\usepackage{amsfonts}   % Font matematici
\usepackage{mathrsfs}   % Scrittura calligrafica matematica
\usepackage{amsthm}     % Teoremi e definizioni

% Pacchetti per grafici
\usepackage{graphicx}   % Inclusione grafica
\usepackage{tikz}       % Disegni e grafici
\usepackage{pgfplots}   % Grafici avanzati
\pgfplotsset{compat=1.18} % Imposta compatibilità per pgfplots

% Pacchetti per la gestione degli indici e riferimenti
\usepackage{hyperref}   % Riferimenti ipertestuali
\hypersetup{
    colorlinks=true,    % Colori per i link
    linkcolor=darkblue, % Colore dei link interni
    citecolor=darkblue, % Colore dei riferimenti alle citazioni
    filecolor=darkblue, % Colore dei link ai file
    urlcolor=darkblue,  % Colore degli URL
    pdftitle={Formulario di Analisi Reale},
    pdfpagemode=UseOutlines
}
\usepackage{tocbibind}  % Include l'indice e la bibliografia nel sommario
\usepackage{fancyhdr}   % Intestazioni e piè di pagina personalizzati
\usepackage{bookmark}   % Gestione segnalibri PDF

% Configurazione per intestazioni e piè di pagina
\setlength{\headheight}{15pt} % Altezza dell'intestazione
\addtolength{\topmargin}{-10pt} % Riduzione del margine superiore

\pagestyle{fancy}
\fancyhf{}  % Pulisce intestazione e piè di pagina
\fancyhead[L]{Formulario di Analisi Reale}  % Intestazione sinistra
\fancyhead[R]{\today}                      % Intestazione destra
\fancyfoot[C]{\thepage}                    % Numero di pagina al centro

% Impostazioni dei margini
\usepackage{geometry}
\geometry{
    left=1.2in,
    right=1.2in,
    top=1in,
    bottom=1.2in
}

% Gestione delle sezioni e sottosezioni
\usepackage{titlesec}
\titleformat{\section}[block]{\large\scshape}{\thesection}{1em}{} % Titoli sezioni maiuscoli
\titleformat{\subsection}[block]{\normalsize\bfseries}{\thesubsection}{1em}{}
\titleformat{\subsubsection}[block]{\normalsize\itshape}{\thesubsubsection}{1em}{}
\titleformat{\paragraph}[runin]{\normalsize\bfseries}{\theparagraph}{1em}{}[:]
\titleformat{\subparagraph}[runin]{\normalsize\itshape}{\thesubparagraph}{1em}{}[:]

% Colore e aspetto
\usepackage{xcolor}
\definecolor{darkblue}{rgb}{0.0, 0.2, 0.6}
\definecolor{gray}{rgb}{0.5, 0.5, 0.5}

% Teoremi e definizioni formali
\newtheoremstyle{mystyle} % Definisci lo stile del teorema
  {10pt} % Spaziatura sopra
  {10pt} % Spaziatura sotto
  {\itshape} % Corpo del teorema in corsivo
  {} % Indentazione del numero
  {\bfseries} % Font del titolo
  {}     % Punteggiatura dopo il titolo
  {\newline} % Spazio dopo il titolo
  {}     % Stile del testo

\theoremstyle{mystyle}
\newtheorem{theorem}{Teorema}[section]
\newtheorem{definition}[theorem]{Definizione}
\newtheorem{lemma}[theorem]{Lemma}
\newtheorem{corollary}[theorem]{Corollario}
\newtheorem{proposition}[theorem]{Proposizione}

% Pacchetti aggiuntivi
\usepackage{enumitem}   % Gestione elenchi personalizzati
\usepackage{multicol}   % Colonne multiple
\usepackage{booktabs}   % Tabelle formattate professionalmente
\usepackage{caption}    % Migliore gestione delle didascalie
\captionsetup{
    labelfont=bf,
    font=small,
    labelsep=colon
}

% Impostazioni di line spacing
\usepackage{setspace}
\onehalfspacing  % Interlinea 1.5 per migliorare la leggibilità

% Informazioni del documento
\title{\textbf{Trigonometria}}
\author{\textit{Oudeys}}
\date{\today}

% Inizio del documento
\begin{document}

\maketitle


\tableofcontents
\newpage

\section{Trigonometria piana}
\subsection{Limitazioni}
\(\lvert \sin x \rvert \leq 1 \)\\
\(\lvert \cos x \rvert \leq 1 \)

\subsection{Teorema di Pitagora}
\(\sin ^2 x + \cos^2 x = 1\)

\subsection{Archi opposti}
\(\sin(-x) = - \sin x \)\\
\(\cos (-x) = \cos x \)\\
\(\tan (-x) = -\tan x \)


\subsection{Archi complementari}
\(\sin \left ( \frac{\pi}{2} -x \right) = \cos x \)\\
\(\cos \left (\frac{\pi}{2}-x  \right) = \sin x \)\\
\(\tan \left ( \frac{\pi}{2} -x\right) = \frac{1}{\tan x} \)

\subsection{Archi supplementari}
\(\sin(\pi - x) = \sin x \)\\
\(\cos (\pi - x) = - \cos x \)\\
\(\tan ( \pi - x) = - \tan x \)

\subsection{Formule di duplicazione}
\(\sin(2x) = 2\sin x \cos x \)\\
\(\cos(2x) = \cos^2 x - \sin^2 x = \begin{cases} 1 - 2 \sin^2 x & \\
2\cos^2 x - 1 \end{cases} \)\\
\(\tan (2x) = \frac{2 \tan x}{1 - \tan^2 x} \)\\
\(\sin^2(x)\cos^2(x) = \frac{1}{4}\sin^2 (2x) = \left( \frac{\sin(2x)}{2}\right)^2 \)

\subsection{Formule parametriche}
\(\forall \, t = \frac{\tan x}{2}\)\\
\(\sin x = \frac{2t}{1 + t^2} \)\\
\(\cos x = \frac{1 - t^2}{1 + t^2} \)\\
\(\tan x = \frac{2t}{1 - t^2} \)


\subsection{Formule di addizione}
\(\sin(x \pm y) = \sin x \cos y \pm \cos x \sin y \)\\
\(\cos(x \pm y) = \cos x \cos y \mp \sin x \sin y \)\\
\(\tan (x \pm y) = \frac{\tan x \pm \tan y}{1 \mp \tan x \cdot \tan y} \)

\subsection{Formule di prostaferesi}
\(\sin x + \sin y = 2 \sin \frac{x+y}{2} \cos \frac{x-y}{2} \)\\
\(cos x + \cos y = 2 \cos \frac{x+y}{2} \cos \)\\
\(\tan x + \tan y = \frac{\sin(x+y)}{\cos x \cos y} \)

\subsection{Formule di Werner}
\(\sin x + \sin y = \frac{1}{2} (\cos(x-y) - \cos(x+y)) \)\\
\(\cos x \cos y = \frac{1}{2} (\cos (x+y) + \cos (x-y)) \)\\
\(\sin x \cos y = \frac{1}{2} (\sin(x+y) + \sin(x-y)) \)


\subsection{Valori per angoli particolari}
\scalebox{1}{
\begin{tabular}{|c|c|c|c|c|}
    \hline
    \( x \) & \( \cos x \) & \( \sin x \) & \( \tan x \) & \( \arctan x \) \\ % Riga di intestazione
    \hline
    \( 0 \)  & \( 1 \)  & \( 0 \)  & \( 0 \)  & \( \nexists \)  \\ % Riga 1
    \hline
    \( \frac{\pi}{12} \)  & \( \frac{\sqrt{6}+\sqrt{2}}{4} \)  & \( \frac{\sqrt{6}-\sqrt{2}}{4} \)   & \( 2-\sqrt{3} \)  & \( 2+\sqrt{3} \) \\ % Riga 2
    \hline
    \( \frac{\pi}{10} \) & \( \frac{\sqrt{10+2\sqrt{5}}}{4} \) & \( \frac{\sqrt{5}-1}{4} \) & \( \frac{\sqrt{25-10\sqrt{5}}}{5} \) & \( \sqrt{5+2\sqrt{5}} \) \\ % Riga 3
    \hline
    \( \frac{\pi}{8} \) & \( \frac{2+\sqrt{2}}{2} \) & \( \frac{\sqrt{2}-\sqrt{2}}{2} \) & \( \sqrt{2} -1 \) & \( \sqrt{2}+1 \) \\ % Riga 4
    \hline
    \( \frac{\pi}{6} \) & \( \frac{\sqrt{3}}{2} \) & \( \frac{1}{2} \) & \( \frac{1}{\sqrt{3}} \) & \( \sqrt{3} \) \\ % Riga 5
    \hline
    \( \frac{\pi}{5} \) & \( \frac{\sqrt{5}+1}{4} \) & \( \frac{\sqrt{10-2\sqrt{5}}}{4} \) & \( \sqrt{5-2\sqrt{5}} \) & \( \sqrt{10+2\sqrt{5}} \) \\ % Riga 6
    \hline
    \( \frac{\pi}{4} \) & \( \frac{\sqrt{2}}{2} \) & \( \frac{\sqrt{2}}{2} \) & \( 1 \) & \( 1 \) \\ % Riga 7
    \hline
    \( \frac{3\pi}{10} \) & \( \frac{\sqrt{10-2\sqrt{5}}}{4} \) & \( \frac{\sqrt{5}+1}{4} \) & \( \sqrt{5+2\sqrt{5}} \) & \( \sqrt{5-2\sqrt{5}} \) \\ % Riga 8
    \hline
    \( \frac{\pi}{3} \) & \( \frac{1}{2} \) & \( \frac{\sqrt{3}}{2} \) & \( \sqrt{3} \) & \( \frac{1}{\sqrt{3}} \) \\ % Riga 9
    \hline
    \( \frac{3\pi}{8} \) & \( \frac{\sqrt{2}-\sqrt{2}}{2} \) & \( \frac{2+\sqrt{2}}{2} \) & \( \sqrt{2}+1 \) & \( \sqrt{2}-1 \) \\ % Riga 10
    \hline
    \( \frac{2\pi}{5} \) & \( \frac{\sqrt{5}-1}{4} \) & \( \frac{\sqrt{10+2\sqrt{5}}}{4} \) & \( \sqrt{25-10\sqrt{5}} \) & \( \frac{\sqrt{5}-1}{4} \) \\ % Riga 11
    \hline
    \( \frac{5\pi}{12} \) & \( \frac{\sqrt{6}-\sqrt{2}}{4} \) & \( \frac{\sqrt{6}+\sqrt{2}}{4} \) & \( \sqrt{3}+2 \) & \( 2-\sqrt{3} \) \\ % Riga 12
    \hline
    \( \frac{\pi}{2} \) & \( 0 \) & \( 1 \) & \( \nexists \) & \( 0 \) \\ % Riga 13
    \hline
    \( \pi \) & \( -1 \) & \( 0 \) & \( 0 \) & \( \nexists \) \\ % Riga 14
    \hline
    \( \frac{3\pi}{2} \) & \( 0 \) & \( -1 \) & \( \nexists \) & \( 0 \) \\ % Riga 15
    \hline
    \( 2\pi \) & \( 1 \) & \( 0 \) & \( 0 \) & \( \nexists \) \\ % Riga 16
    \hline
\end{tabular}
}

\section{Trigonometria sferica}

\end{document}